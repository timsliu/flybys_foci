\documentclass[12pt]{article} % use larger type; default would be 10pt
\usepackage[utf8]{inputenc} % set input encoding (not needed with XeLaTeX)

\usepackage{geometry, ulem, soul, color, graphicx, hyperref, array, caption, titling} % to change the page dimensions
\geometry{letterpaper}
\usepackage[dvipsnames]{xcolor}


 %put box around figure captions
\makeatletter
\long\def\@makecaption#1#2{%
  \vskip\abovecaptionskip
  \sbox\@tempboxa{\fbox{#1: #2}}%
  \ifdim \wd\@tempboxa >\hsize
    \fbox{\parbox{\dimexpr\linewidth-2\fboxsep-2\fboxrule}{#1: #2}}\par
  \else
    \global \@minipagefalse
    \hb@xt@\hsize{\hfil\box\@tempboxa\hfil}%
  \fi
  \vskip\belowcaptionskip}
\makeatother

%define a subtitle
\usepackage{titling}
\newcommand{\subtitle}[1]{%
  \posttitle{%
    \par\end{center}
    \begin{center}\large#1\end{center}
    \vskip0.5em}%
}


\setlength\parindent{0pt}
\setlength\parskip{8pt}
\usepackage{float}


\geometry{margin = .8in}

\hypersetup{  %set up url
    colorlinks=true,
    linkcolor=blue,
    filecolor=magenta,      
    urlcolor=cyan,
}

%IHC logo
\title{\textbf{\textcolor{Red}{Bechtel Occupancy Survey}}}
\subtitle{Things We Wish We Had Known 3 Years Ago}

\begin{document}
\maketitle
\tableofcontents
\newpage

\section{Introduction}

During spring term of 2017, two students (the authors) surveyed the undergraduate population to learn why students move off campus. The goal of the Bechtel Occupancy Survey was to understand why different cohorts move off campus and if students will live in Bechtel. The survey was sent out on May 22nd. The authors distributed roughly 15 pounds of chocolate (paid for by the IHC) to off campus students to boost the off campus response rate. \textbf{578 students responded to the survey, representing 60.1\%} of the 961 students enrolled as of add day spring term.

The survey asked students:

\begin{enumerate}
	\item Demographic info (matriculation year, primary house affiliation).
	\item Where did you live for each year you have been enrolled.
	\item Single most important reason for living off campus (if applicable).
	\item All other reasons for living off campus (if applicable).
	\item Would you have been happier living in an unaffiliated single (i.e. Marks-Braun) during freshman year over the house you rotated into? 
	\item Would you have been happier living in an unaffiliated single (i.e. Marks-Braun) during freshman year over your TOP house choice?
	\item Which years would you like to (or would have liked to) reside physically on campus?
	\item If Bechtel is unaffiliated singles, would you choose to live in Bechtel over your current housing situation?
	\item If Bechtel is arranged into house-affiliated off campus alleys (OCAs), would you choose to live in Bechtel over your current housing situation?
	\item How frequently do you visit any of the 8 houses?
\end{enumerate}

This document summarizes the results of the survey.
\newpage
\section{Notes}
Below are several important notes pertaining to how data was collected and analyzed:

\begin{itemize}
\item This survey was sent to the ug-list mailing list and the authors sought responses from all students. The authors made special efforts to boost the response rate from students unaffiliated with houses or living off campus. The response rate for students living off campus appears to be similar to the overall response rate.
\item All error bars denote 1 standard deviation of error. These bars only account for random sampling error. Systematic biases are not bounded by these errors. It is generally assumed that respondents in a cohort are representative of a cohort, though this may not always be true.
\item Respondents were asked where they lived for each year they were at Caltech. If respondents lived off campus, they were then asked why for each year. In the analysis, these answers were summed, effectively giving multiple years of data to work with. This allowed the authors to gather data from not just students currently in a particular residence but any student who recently lived there. This also means that a student living in, for example, Marks for three years would count as three responses. This practice is effectively the same as repeating a survey asking students why they moved off a particular year and aggregating survey results from several years together.
\item For responses sorted by matriculation rate, ``2012" denotes students who matriculated in 2012 or earlier. 
\item A total of 7 students not currently at Caltech (likely from leaves of absence) responded to the survey. These students were not counted in cohorts of students currently living in certain residences, but were otherwise counted. 
\item Several questions in the survey directly address proposed Bechtel schemes. The authors intentionally decided not to ask questions related to themed housing or new Houses. The authors felt that these two proposals were too nebulous to ask specific questions.
\item The number of responses for question 6 is fewer than the number of responses for question 5 because the question was originally misstated in the survey. Response for the mistakenly worded question were thrown out.
\item Students were allowed to select ``Other" for their reasons they chose to move off campus. Several responses in the ``Other" category were manually sorted into one of the given choices if they matched.
\item Once Bechtel is completed, the Del Mar quads and House Off Campus Alleys will no longer be available to undergrads. Several graphs sort response by ``offkeep" - off campus residences that will be kept - and ``offgone" - off campus residences that will be removed (Del Mars and OCAs).
\item Numbers above each bar denote the total number of responses.
\end{itemize}

\newpage
\section{Demographic Information}
\subsection{Demographic Info by Year}

\begin{figure}[H]
	\makebox[\textwidth][c]{\includegraphics[width=6in]{resp_year.png}}
	\caption{Responses sorted by matriculation year.}
\end{figure}

\begin{figure}[H]
	\makebox[\textwidth][c]{\includegraphics[width=6in]{rel_resp_y.png}}
	\caption{Response rate sorted by matriculation year. Data on number of enrolled students by matriculation year 		provided by the Registrar's Office.}
\end{figure}
\subsection{Demographic Info by House Affiliation}

\begin{figure}[H]
	\makebox[\textwidth][c]{\includegraphics[width=6.5in]{resp_house.png}}
	\caption{Responses sorted by primary house affiliation.}
\end{figure}

\begin{figure}[H]
	\makebox[\textwidth][c]{\includegraphics[width=6.5in]{rel_resp_h.png}}
	\caption{Response rate sorted by primary house affiliation. Data on house affiliations estimated from data on
	Caltech student directory (Donut).}
\end{figure}

\subsection{Demographic Info by Residence}


\begin{figure}[H]
	\makebox[\textwidth][c]{\includegraphics[width=8in]{rel_resp_r.png}}
	\caption{Response rate sorted by residence. Data on number of students living in houses \emph{estimated} from house capacity data supplied by housing office. More detailed residence response rate will become available, pending data from housing office.}
\end{figure}


\newpage
\section{Most Important Reasons for Living Off Campus}

For this section, respondents were asked to select the \textbf{single most important reason} they chose to live off campus.
``Off campus'' is defined as any residence that is not one of the 8 houses. Students who lived in Avery via the external lottery were treated as students living in Avery and not off campus. \subsection{Sorted by Year}
\vspace{-5mm}
\begin{figure}[H]

\makebox[\textwidth][c]{\includegraphics[width=6.5in]{y_off_year.png}}
\vspace{-14mm}
\begin{center}
\begin{tabular}{|m{6 cm}|m{1.7 cm}|m{1.7 cm}|m{1.7 cm} |}
\hline
\multicolumn{4}{|c|}{Most Important Reason for Living Off}\\ \hline
& EYear2& EYear3& EYear4\\ \hline
Board Cost   &  0.084 &  0.167 &  0.127\\ \hline
Board Quality   &  0.100 &  0.141 &  0.145\\ \hline
Internal Number (kicked off)   &  0.404 &  0.173 &  0.055\\ \hline
Friends Off   &  0.084 &  0.077 &  0.073\\ \hline
External Number   &  0.100 &  0.071 &  0.036\\ \hline
Amenities (single, living space)   &  0.060 &  0.122 &  0.200\\ \hline
House Social Environment   &  0.048 &  0.090 &  0.173\\ \hline
Cost of Housing   &  0.036 &  0.045 &  0.064\\ \hline
Other   &  0.084 &  0.115 &  0.127\\ \hline
\end{tabular}
\end{center}
\vspace{-5mm}
\caption{Single most important reason for living off campus, sorted by ``Effective Year." Effective year is the reported year living off minus matriculation year. EYear2 generally corresponds to sophomores, EYear3 to juniors, and EYear4 to seniors.}
\end{figure}

\vspace{-20mm}
\subsection{Sorted by Residence}
\begin{figure}[H]
\makebox[\textwidth][c]{\includegraphics[width=6.5in]{y_off_res.png}}	
\begin{center}
\vspace{-10mm}
\begin{tabular}{|m{6 cm}|m{1.7 cm}|m{1.7 cm}|m{1.7 cm}|m{1.7 cm}|m{1.8 cm} |}
\hline
\multicolumn{6}{|c|}{Most Important Reason for Living Off}\\ \hline
& LocMB& LocDM& LocCC& LocOCA& LocOffOff\\ \hline
Board Cost   &  0.000 &  0.133 &  0.069 &  0.180 &  0.148\\ \hline
Board Quality   &  0.045 &  0.133 &  0.108 &  0.117 &  0.159\\ \hline
Internal Number \newline (kicked off)   &  0.351 &  0.156 &  0.392 &  0.234 &  0.148\\ \hline
Friends Off   &  0.000 &  0.022 &  0.118 &  0.156 &  0.068\\ \hline
External Number   &  0.171 &  0.333 &  0.029 &  0.031 &  0.000\\ \hline
Amenities \newline(single, living space)   &  0.189 &  0.089 &  0.137 &  0.070 &  0.085\\ \hline
House Social\newline Environment   &  0.081 &  0.089 &  0.069 &  0.148 &  0.051\\ \hline
Cost of Housing   &  0.000 &  0.000 &  0.000 &  0.008 &  0.165\\ \hline
Other   &  0.162 &  0.044 &  0.078 &  0.055 &  0.176\\ \hline
\end{tabular}
\end{center}
\vspace{-5mm}
	\caption{Single most important reason for living off campus, sorted by residence. 
	\newline\textbf{LocMB} - Marks Braun
	\newline\textbf{LocDM} - Del Mar
	\newline\textbf{LocCC} - Chesters/Catalinas,
	\newline\textbf{LocOCA} - Off Campus Alleys
	\newline\textbf{LocOffOff} - not Caltech affiliated housing.}
\end{figure}
\vspace{-20mm}


\subsection{Sorted By Future Off Campus Options}
\vspace{-5mm}
\begin{figure}[H]
	\makebox[\textwidth][c]{\includegraphics[width=7in]{y_off_fut2.png}}	
\begin{center}
\begin{tabular}{|m{4 cm}|m{2.6 cm}|m{2.6 cm}|m{2.6 cm} |}
\hline
\multicolumn{4}{|c|}{Most Important Reason for Living Off}\\ \hline
& LocMB& LocOffDefunct& LocOffOff\\ \hline
Board Cost   &  0.000 &  0.131 &  0.148\\ \hline
Board Quality   &  0.045 &  0.116 &  0.159\\ \hline
Internal Num.\\ (kicked off)   &  0.351 &  0.280 &  0.148\\ \hline
Friends Off   &  0.000 &  0.120 &  0.068\\ \hline
External Num.   &  0.171 &  0.080 &  0.000\\ \hline
Amenities
(single, living space)   &  0.189 &  0.098 &  0.085\\ \hline
House Social
Environment   &  0.081 &  0.109 &  0.051\\ \hline
Cost of Housing   &  0.000 &  0.004 &  0.165\\ \hline
Other   &  0.162 &  0.062 &  0.176\\ \hline
\end{tabular}
\end{center}
	\caption{Single most important reason for living off campus, sorted by future available off campus residences. OffDefunct denotes residences that will no longer be available when Bechtel is opened (Chesters/cats, Del Mar, OCAs.}

	
\end{figure}

\subsection{Sorted By Rotated House}

\begin{figure}[H]
	\makebox[\textwidth][c]{\includegraphics[width=8in]{y_off_house.png}}
	
\begin{center}
\begin{tabular}{|m{4 cm}|m{1.3 cm}|m{1.3 cm}|m{1.3 cm}|m{1.3 cm}|m{1.3 cm}|m{1.3 cm}|m{1.3 cm}|m{1.3 cm} |}
\hline
\multicolumn{9}{|c|}{Most Important Reason for Living Off}\\ \hline
& Avery&Blacker&Dabney&Fleming&Lloyd&Page&Ricketts&Ruddock\\ \hline
Board Cost   &  0.127 &  0.103 &  0.071 &  0.000 &  0.175 &  0.218 &  0.183 &  0.025\\ \hline
Board Quality   &  0.063 &  0.121 &  0.071 &  0.119 &  0.206 &  0.145 &  0.150 &  0.076\\ \hline
Internal (kicked off)   &  0.190 &  0.250 &  0.476 &  0.167 &  0.222 &  0.055 &  0.367 &  0.190\\ \hline
Friends Off   &  0.079 &  0.103 &  0.036 &  0.143 &  0.032 &  0.109 &  0.067 &  0.089\\ \hline
External Number   &  0.095 &  0.060 &  0.083 &  0.048 &  0.095 &  0.036 &  0.050 &  0.101\\ \hline
Amenities \newline(single, living space)   &  0.063 &  0.034 &  0.083 &  0.214 &  0.111 &  0.164 &  0.033 &  0.266\\ \hline
House Social Environment   &  0.143 &  0.095 &  0.024 &  0.167 &  0.032 &  0.127 &  0.017 &  0.114\\ \hline
Cost of Housing   &  0.127 &  0.095 &  0.024 &  0.000 &  0.000 &  0.036 &  0.067 &  0.038\\ \hline
Other   &  0.111 &  0.138 &  0.131 &  0.143 &  0.127 &  0.109 &  0.067 &  0.101\\ \hline
\end{tabular}
\end{center}
	\caption{Single most important reason for living off campus, sorted by rotated house.}
\end{figure}



\newpage
\section{All Reasons Considered for Living Off Campus}
Survey respondents were asked to identify the single most important factor for living off campus (previous section). Respondents were also asked for any other major factors that were considered. This section sums the two responses and summarizes all the reasons respondents listed for moving off campus.

\subsection{Sorted by Residence}

\begin{figure}[H]
	\makebox[\textwidth][c]{\includegraphics[width=7.5in]{all_reasons_res.png}}
	\caption{All factors considered when moving off, sorted by residence. The decimals denote the fraction of respondents in each column that reported each reason for moving off campus.}

\end{figure}

\subsection{Sorted by Future Existing Residences}

\vspace{5 mm}
\begin{figure}[H]
	\makebox[\textwidth][c]{\includegraphics[width=7.5in]{all_reasons_fut2.png}}
	\vspace{5mm}
	\caption{All factors considered when moving off, sorted by existing off campus options that will be available when Bechtel is completed. Marks and Braun fall under the ``Off - Options Remaining" category. House OCAs,  the Del Mar quads, and 150 S. Chester belong in the ``Off - Options Removed" category. The decimals denote the fraction of respondents in each column that reported each reason for moving off campus.}

\end{figure}
\newpage
\section{Freshmen in Houses}

The survey included the following two questions:
\vspace{-4mm}
\begin{itemize}
\item Would you have been happier living in an unaffiliated single (i.e. Marks-Braun) during freshman year over the house you rotated into? 
\vspace{-2mm}
\item Would you have been happier living in an unaffiliated single (i.e. Marks-Braun) during freshman year over 	your TOP house choice?
\end{itemize}
\vspace{-4mm}

These questions try to gauge if students feel they \textbf{would have been better off opting out of the House System during their freshmen year.} The first question pertains to the house students actually rotated into. The second question attempts to answer the same question under ideal circumstances (if everyone was placed in their top choice house). \textbf{8.3\% of students} felt they would have been happier their freshmen year had they opted out of the house they rotated into. \textbf{5.3\% of students} felt they would have been happier their freshmen year had they lived in an unaffiliated single over their top choice. 
\vspace{-3mm}
\begin{figure}[H]
	\makebox[\textwidth][c]{\includegraphics[width=6.5in]{rot_single_house.png}}
	\begin{center}
	\vspace{-5mm}
\begin{tabular}{|m{1.3 cm}|m{1.3 cm}|m{1.3 cm}|m{1.3 cm}|m{1.3 cm}|m{1.3 cm}|m{1.3 cm}|m{1.3 cm}|m{1.3 cm} |}
\hline
\multicolumn{9}{|c|}{Unaffiliated Single over Rotated House}\\ \hline
& avery& blacker& dabney& fleming& lloyd& page& ricketts& ruddock\\ \hline
Yes   &  0.148 &  0.071 &  0.088 &  0.016 &  0.033 &  0.171 &  0.018 &  0.079\\ \hline
No   &  0.852 &  0.929 &  0.912 &  0.984 &  0.967 &  0.829 &  0.982 &  0.921\\ \hline
\end{tabular}
\end{center}
	\caption{Happier in an unaffiliated single over the house you rotated into, sorted by rotated house.}
\end{figure}



\begin{figure}[H]
	\makebox[\textwidth][c]{\includegraphics[width=8in]{top_single_house.png}}
	\begin{center}
\begin{tabular}{|m{1.3 cm}|m{1.3 cm}|m{1.3 cm}|m{1.3 cm}|m{1.3 cm}|m{1.3 cm}|m{1.3 cm}|m{1.3 cm}|m{1.3 cm} |}
\hline
\multicolumn{9}{|c|}{Unaffiliated Single Over Top Choice}\\ \hline
& avery& blacker& dabney& fleming& lloyd& page& ricketts& ruddock\\ \hline
Yes   &  0.083 &  0.044 &  0.049 &  0.019 &  0.037 &  0.079 &  0.023 &  0.062\\ \hline
No   &  0.917 &  0.956 &  0.951 &  0.981 &  0.963 &  0.921 &  0.977 &  0.938\\ \hline
\end{tabular}
\end{center}
	\caption{Happier in an unaffiliated single over \textbf{your top choice}, sorted by rotated house.}
\end{figure}

\subsection{Unaffiliated Single over Rotated House - Sorted by Residence}
\begin{figure}[H]
	\makebox[\textwidth][c]{\includegraphics[width=7.5in]{rot_single_res.png}}
	\begin{center}
\begin{tabular}{|m{1.3 cm}|m{1.3 cm}|m{1.8 cm}|m{2.3 cm}|m{1.3 cm}|m{1.3 cm}|m{1.3 cm}|m{1.3 cm} |}
\hline
\multicolumn{8}{|c|}{Unaffiliated Single Over Rotated House}\\ \hline
& house& nonhouse& chestercats& delmar& mb& oca& offoff\\ \hline
Yes   &  0.055 &  0.143 &  0.216 &  0.133 &  0.278 &  0.089 &  0.054\\ \hline
No   &  0.945 &  0.857 &  0.784 &  0.867 &  0.722 &  0.911 &  0.946\\ \hline
\end{tabular}
\end{center}
	\caption{Happier in an unaffiliated single over the house you rotated into, sorted by residence.}
\end{figure}

\subsection{Unaffiliated Single over Rotated House - Sorted by Year}
\begin{figure}[H]
	\makebox[\textwidth][c]{\includegraphics[width=7.5in]{rot_single_year.png}}
	\begin{center}
\begin{tabular}{|m{1.3 cm}|m{1.3 cm}|m{1.3 cm}|m{1.3 cm}|m{1.3 cm}|m{1.3 cm} |}
\hline
\multicolumn{6}{|c|}{Unaffiliated Single Over Rotated House}\\ \hline
& 2012& 2013& 2014& 2015& 2016\\ \hline
Yes   &  0.133 &  0.108 &  0.106 &  0.048 &  0.069\\ \hline
No   &  0.867 &  0.892 &  0.894 &  0.952 &  0.931\\ \hline
\end{tabular}
\end{center}
	\caption{Happier in an unaffiliated single over the house you rotated into, sorted by matriculation year.}
\end{figure}
\newpage
\section{Living in Bechtel}
This set of questions asks respondents if they prefer to live in Bechtel over where they currently live. Respondents were given two scenarios - Bechtel as OCAs and Bechtel as unaffiliated housing. The survey asked respondents to assume that students would be on a board plan if they lived in Bechtel.
\subsection{Sorted by Residence}

\begin{figure}[H]
	\makebox[\textwidth][c]{\includegraphics[width=7in]{boca_res.png}}
	\begin{center}
\begin{tabular}{|m{1.3 cm}|m{1.7 cm}|m{2.0 cm}|m{2.3 cm}|m{1.3 cm}|m{1.3 cm}|m{1.3 cm}|m{1.3 cm} |}
\hline
\multicolumn{8}{|c|}{Bechtel OCA Over Where you Currently Live?}\\ \hline
& house& nonhouse& chestercats& delmar& mb& oca& offoff\\ \hline
Yes   &  0.296 &  0.360 &  0.622 &  0.400 &  0.389 &  0.222 &  0.268\\ \hline
No   &  0.704 &  0.640 &  0.378 &  0.600 &  0.611 &  0.778 &  0.732\\ \hline

\end{tabular}
\end{center}
	\caption{Would you live in a Bechtel OCA over where you currently live?}

\end{figure}

\vspace{-7mm}

\begin{figure}[H]
	\makebox[\textwidth][c]{\includegraphics[width=7.5in]{bunaf_res.png}}
	\begin{center}
\begin{tabular}{|m{1.3 cm}|m{1.3 cm}|m{2 cm}|m{2.5 cm}|m{1.3 cm}|m{1.3 cm}|m{1.3 cm}|m{1.3 cm} |}
\hline
\multicolumn{8}{|c|}{Bechtel Unaffiliated Over Where you Currently Live?}\\ \hline
& house& nonhouse& chestercats& delmar& mb& oca& offoff\\ \hline
Yes   &  0.154 &  0.265 &  0.541 &  0.133 &  0.333 &  0.089 &  0.214\\ \hline
No   &  0.846 &  0.735 &  0.459 &  0.867 &  0.667 &  0.911 &  0.786\\ \hline
\end{tabular}
\end{center}
	\caption{Would you live in Bechtel as unaffiliated over where you currently live?}
\end{figure}

\subsection{Sorted by Future Off Campus Options}
\vspace{-5mm}
\begin{figure}[H]
	\makebox[\textwidth][c]{\includegraphics[width=7.5in]{boca_fut2.png}}
	\vspace{-13mm}
\begin{center}
\begin{tabular}{|m{1.3 cm}|m{2.3 cm}|m{2.3 cm}|m{2.3 cm} |}
\hline
\multicolumn{4}{|c|}{Bechtel OCA over Where you Currently Live?}\\ \hline
& mb& offdefunct& offoff\\ \hline
Yes   &  0.389 &  0.402 &  0.268\\ \hline
No   &  0.611 &  0.598 &  0.732\\ \hline
\end{tabular}
\end{center}
	\caption{Would you live in a Bechtel OCA over where you currently live?
	\newline \textbf{offkeep - } Off campus options that will remain (Marks, Braun). 	
	\newline\textbf{offgone - } Off campus options being removed (OCA, Del Mar, Chesters)}
\end{figure}

\vspace{-5mm}

\begin{figure}[H]
	\makebox[\textwidth][c]{\includegraphics[width=7.5in]{bunaf_fut2.png}}
	\vspace{-13mm}
\begin{center}
\begin{tabular}{|m{1.3 cm}|m{2.3 cm}|m{2.3 cm}|m{2.3 cm} |}
\hline
\multicolumn{4}{|c|}{Bechtel Unaffiliated over Where you Currently Live?}\\ \hline
& mb& offdefunct& offoff\\ \hline
Yes   &  0.333 &  0.268 &  0.214\\ \hline
No   &  0.667 &  0.732 &  0.786\\ \hline
\end{tabular}
\end{center}
	\caption{Would you live in Bechtel unaffiliated over where you currently live?
	\newline \textbf{offkeep - } Off campus options that will remain (Marks, Braun). 	
	\newline\textbf{offgone - } Off campus options being removed (OCA, Del Mar, Chesters)}
\end{figure}
\newpage

\section{Years Prefer Living On}
Respondents were asked which years - sophomore, junior, and senior - they would prefer to physically reside on campus. All respondents, regardless of current academic year, answered this question. Respondents were asked to assume that physically residing on campus would force them to be on board.

\subsection{Sorted by Matriculation Year}
\begin{figure}[H]
	\makebox[\textwidth][c]{\includegraphics[width=7.5in]{live_on_year.png}}
	\vspace{-13mm}
	\begin{center}
\begin{tabular}{|m{2 cm}|m{1.3 cm}|m{1.3 cm}|m{1.3 cm}|m{1.3 cm}|m{1.3 cm} |}
\hline
\multicolumn{6}{|c|}{Which Years Would you Like to Live on Campus?}\\ \hline
& 2012& 2013& 2014& 2015& 2016\\ \hline
Sophomore   &  0.633 &  0.725 &  0.631 &  0.733 &  0.792\\ \hline
Junior   &  0.700 &  0.735 &  0.589 &  0.685 &  0.799\\ \hline
Senior   &  0.767 &  0.647 &  0.645 &  0.733 &  0.849\\ \hline
\end{tabular}
\end{center}
	\caption{Years respondent prefers to physically reside on campus, sorted by respondent matriculation year.}
\end{figure}

\subsection{Sorted by Rotated House}
\begin{figure}[H]
	\makebox[\textwidth][c]{\includegraphics[width=8in]{live_on_house_r.png}}
	\begin{center}
\begin{tabular}{|m{2 cm}|m{1.3 cm}|m{1.3 cm}|m{1.3 cm}|m{1.3 cm}|m{1.3 cm}|m{1.3 cm}|m{1.3 cm}|m{1.3 cm} |}
\hline
\multicolumn{9}{|c|}{Which Years Would you Like to Live on Campus?}\\ \hline
& avery& blacker& dabney& fleming& lloyd& page& ricketts& ruddock\\ \hline
Sophomore   &  0.807 &  0.600 &  0.735 &  0.871 &  0.689 &  0.671 &  0.545 &  0.787\\ \hline
Junior   &  0.716 &  0.706 &  0.838 &  0.855 &  0.557 &  0.600 &  0.673 &  0.674\\ \hline
Senior   &  0.716 &  0.765 &  0.838 &  0.726 &  0.541 &  0.671 &  0.836 &  0.742\\ \hline
\end{tabular}
\end{center}
	\caption{Years respondents prefer to physically reside on campus, sorted by rotated house.}
	
\end{figure}

\subsection{Sorted by Residence}

\begin{figure}[H]
	\makebox[\textwidth][c]{\includegraphics[width=8in]{live_on_res.png}}
	\begin{center}
\begin{tabular}{|m{2 cm}|m{1.7 cm}|m{1.7 cm}|m{2 cm}|m{1.7 cm}|m{1.7 cm}|m{1.7 cm}|m{1.7 cm} |}
\hline
\multicolumn{8}{|c|}{Which Years Would you Like to Live on Campus?}\\ \hline
& house& nonhouse& chestercats& delmar& mb& oca& offoff\\ \hline
Sophomore   &  0.796 &  0.556 &  0.622 &  0.333 &  0.722 &  0.533 &  0.482\\ \hline
Junior   &  0.801 &  0.492 &  0.514 &  0.400 &  0.750 &  0.489 &  0.339\\ \hline
Senior   &  0.835 &  0.508 &  0.595 &  0.467 &  0.750 &  0.400 &  0.393\\ \hline
\end{tabular}
\end{center}
	\caption{Years respondents prefer to physically reside on campus, sorted by current residence}
\end{figure}
\newpage

\section{Visiting Houses}
This section summarizes how often students living off campus self-report visiting any one of the 8 houses. This indicator is a proxy for how important the house system is for students living off campus.

\subsection{Sorted by Residence}
\vspace{-5mm}
\begin{figure}[H]
	\makebox[\textwidth][c]{\includegraphics[width=8in]{off_visit.png}}
	\begin{center}
\begin{tabular}{|m{6 cm}|m{1.7 cm}|m{2 cm}|m{1.3 cm}|m{1.3 cm}|m{1.3 cm}|m{1.3 cm} |}
\hline
\multicolumn{7}{|c|}{How Often Do you Visit any of the 8 Houses?}\\ \hline
& nonhouse& chestercats& delmar& mb& oca& offoff\\ \hline
Every Day   &  0.333 &  0.297 &  0.200 &  0.556 &  0.311 &  0.268\\ \hline
Most day each week   &  0.265 &  0.297 &  0.200 &  0.139 &  0.311 &  0.304\\ \hline
Several days each week   &  0.111 &  0.081 &  0.133 &  0.083 &  0.133 &  0.125\\ \hline
About once a week   &  0.095 &  0.054 &  0.267 &  0.028 &  0.089 &  0.125\\ \hline
Less than once a week   &  0.196 &  0.270 &  0.200 &  0.194 &  0.156 &  0.179\\ \hline
\end{tabular}
\end{center}
	\caption{How often do you visit any of the 8 houses? Notice that residents of Marks and Braun are most likely
	to report visiting a house every day.}

\end{figure}

\subsection{Sorted by Matriculation Year}
\vspace{-5mm}
\begin{figure}[H]
	\makebox[\textwidth][c]{\includegraphics[width=8in]{off_visit_yr.png}}
\begin{center}
\begin{tabular}{|m{6 cm}|m{1.7 cm}|m{1.7 cm}|m{1.7 cm}|m{1.7 cm} |}
\hline
\multicolumn{5}{|c|}{How Often Do you Visit any of the 8 Houses?}\\ \hline
& 2012& 2013& 2014& 2015\\ \hline
Every Day   &  0.294 &  0.255 &  0.294 &  0.408\\ \hline
Most days each week   &  0.176 &  0.319 &  0.255 &  0.268\\ \hline
Several days each week   &  0.000 &  0.085 &  0.098 &  0.155\\ \hline
About once a week   &  0.176 &  0.064 &  0.157 &  0.056\\ \hline
Less than once a week   &  0.353 &  0.277 &  0.196 &  0.113\\ \hline
\end{tabular}
\end{center}
	\caption{How often do you visit any of the 8 houses? Responses are from students living off campus and sorted by matriculation year.}

\end{figure}

\newpage
\section{Conclusions}
Several observations from the data:

\begin{itemize}
\item The response rate of off campus students was roughly the same as the overall response rate.
\item \textbf{Internal lottery numbers} are the most significant reason students move off their \textbf{sophomore year}. The importance of internal lottery numbers in deciding to move off declines as students progress through Caltech.
\item The \textbf{quality and cost of board become increasingly important} factors to students who choose to move off campus.
\item The \textbf{cost of housing and board} are especially important for students who move \textbf{off off}.
\item The large majority of students (90\%) do not believe they would have been happier not living in a house their freshman year.
\item \textbf{Off campus alleys} are generally a more popular option than Bechtel unaffiliated.
\item Students living off campus generally visit the houses less frequently as they progress through Caltech.
\item Students who live in off campus options that will be removed prioritized similar factors as students who live off off. The exception is that students who live off off are more sensitive to the cost of housing.
\end{itemize}

Several of the features of Bechtel have already been decided and cannot be easily modified (number of kitchens, layout, etc.). Aspects of Bechtel that can still be changed include the social environment of the residence, board cost, and who lives there. Bechtel is made of a mix of suites of varying sizes and some singles. This layout may make integrating Bechtel into the external unaffiliated lottery system complicated.


\end{document}












