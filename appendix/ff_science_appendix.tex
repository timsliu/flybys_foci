\documentclass[12pt]{article} % use larger type; default would be 10pt

%packages
\usepackage[utf8]{inputenc} % set input encoding (not needed with XeLaTeX)
\usepackage{fancyhdr}
\usepackage{float}
\usepackage{geometry}
\usepackage{ulem}
\usepackage{soul}
\usepackage{color}
\usepackage{graphicx}
\usepackage{hyperref}
\usepackage{array}
\usepackage{caption}
\usepackage{titling}
\usepackage{enumerate} 
\usepackage[dvipsnames]{xcolor}
\usepackage{amsmath}
\usepackage{amssymb}
\usepackage[compact]{titlesec}


 %put box around figure captions
\makeatletter
\long\def\@makecaption#1#2{%
  \vskip\abovecaptionskip
  \sbox\@tempboxa{\fbox{#1: #2}}%
  \ifdim \wd\@tempboxa >\hsize
    \fbox{\parbox{\dimexpr\linewidth-2\fboxsep-2\fboxrule}{#1: #2}}\par
  \else
    \global \@minipagefalse
    \hb@xt@\hsize{\hfil\box\@tempboxa\hfil}%
  \fi
  \vskip\belowcaptionskip}
\makeatother

%reduce space between sections
\titlespacing{\section}{0pt}{*1}{*0}
\titlespacing{\subsection}{0pt}{*1}{*0}
\titlespacing{\subsubsection}{0pt}{*0}{*0}


%no indent and modify distance between paragraphs
\setlength\parindent{0pt}
\setlength\parskip{12pt}

%set margins and line spacing
\geometry{margin=1in}
\linespread{1.2}
\geometry{letterpaper}

%math operators
\DeclareMathOperator{\E}{\mathbb{E}}

%set up header and page numbering
\pagestyle{fancy}
\lhead{Scientific Appendix}
\rhead{Timothy Liu}
\pagenumbering{arabic}

\hypersetup{  %set up url
    colorlinks=true,
    linkcolor=blue,
    filecolor=magenta,      
    urlcolor=cyan,
}



\title{Flybys and Foci Scientific Appendix}
\author{Timothy Liu}

\begin{document}

\maketitle

Throughout the story I have done my best to keep the plot scientifically accurate and plausible. This appendix gives a more thorough explanation of some parts of the story that the reader may find interesting. I have written the appendix assuming the reader has some basic knowledge of physics, to the level that can be found on Wikipedia.

The appendix is organized into four sections. Section 1 explains the flight path through space \textit{Einstein} takes and why it makes an unintuitive flyby of the sun to reach deep space. The second section details the gravitational lensing effect and calculates how far away the sun's focal point is. Section 3 gives a rough estimate of how efficient the \textit{Sheridan} drive is and how it stacks up against current engines. Section 4 is the culmination of the first three sections and calculates how long the flight to the sun's focal point is.

\tableofcontents

\section{Diving Towards the Sun}

In Chapter 8, \textit{Einstein} is put on a trajectory that passes close to the sun before performing its main burn to exit the solar system. This may seem unintuitive - the spacecraft first heads towards the sun to make it to a point very distant to the sun. \textit{Einstein} is taking advantage of the Oberth Effect, where the most efficient place to perform an engine burn is deep in a gravity well.

In rocketry a common unit for describing the effort needed to travel between points in space is delta v, also expressed as $\Delta v$. Delta v is the change in velocity needed to modify a spacecraft's orbit. For example, to go from low earth orbit to a trajectory that escapes from earth's gravity takes a change in velocity (or delta v) of 3.2 km/s. The total velocity change that a spacecraft can perform is called the delta v budget. This section explains how \textit{Einstein}'s trajectory in the story maximizes its delta v budget to reach the focal point as quickly as possible.

The total energy of a body in orbit is the sum of the kinetic and potential energy:

$$ E = \frac{1}{2} mv^2 - \frac{mMG}{r} = -\frac{mMG}{2a}$$

Where:

$m$ = spacecraft mass \\
$v$ = velocity \\
$M$ = solar mass\\
$G$ = universal gravitational constant\\
$r$ = distance to the sun\\
$a$ = semi-major axis\\

This expression is commonly divided by mass to get the \textit{specific orbital energy}:

$$ = \frac{1}{2} v^2 - \frac{MG}{r} = -\frac{MG}{2a}$$

For \textit{Einstein} the higher the specific orbital energy the faster it will reach the suns focal point. A more tangible measure is $v_{\infty}$, which is the spacecraft's velocity when it is so far away from the sun that its potential energy is negligible. As a spacecraft climbs out of the sun's gravity well it sheds velocity and the potential energy goes to 0 (in orbital mechanics potential energy is 0 at infinity and increasingly negative closer to the sun).

$$\frac{1}{2}v_{\infty}^{2} = \frac{1}{2} v^2 - \frac{MG}{r} = -\frac{MG}{2a}$$
$$v_{\infty} = \sqrt{v^2-\frac{2MG}{r}}$$

Where:

$r$ = spacecraft distance to the sun\\
$v$ = velocity at the given $r$\\

We can use this to calculate $v_{\infty}$ for \textit{Einstein} if it had burned directly on a hyperbolic escape trajectory and compare it to $v_{\infty}$ from its trajectory that took it close to the sun.

We can use this to calculate $v_{\infty}$ for \textit{Einstein} if it had burned directly on a hyperbolic escape trajectory and compare it to $v_{\infty}$ from its actual trajectory that took it close to the sun.

\subsection{Scenario 1: Direct escape burn}

Assume that \textit{Einstein} begins in a heliocentric, circular orbit the same distance from the sun as Jupiter. In the story \textbf{Einstein} must first escape from Jupiter's gravity well but for simplicity we'll assume this doesn't require much $\Delta$ v.

$$\frac{1}{2}v_{\infty}^{2} = \frac{1}{2} (v_0 + \Delta v)^2 - \frac{MG}{r_0}$$
$$v_{\infty} = \sqrt{(v_0 + \Delta v)^2-\frac{2MG}{r_0}}$$

Where:
$v_0$: starting velocity\\
$\Delta v$: change in velocity supplied by \textit{Sheridan} drive\\
$r_0$: starting distance from the sun - Jupiter's orbit


\subsection{Scenario 2: Oberth effect}

Again we assume that \textit{Einstein} begins in a heliocentric, circular orbit the same distance from the sun as Jupiter. \textit{Einstein} then performs two burns - one retrograde (opposite the direction of orbit) to bring down the perihelion (the point in the orbit nearest the sun) and a second burn near the sun to escape the solar system. In the story the first burn is combined with the maneuver to escape Jupiter, again is taking advantage of the Oberth effect!

The hyperbolic excess velocity ($\Delta v_{\infty}$) following the second burn is:

$$v_{\infty} = \sqrt{(v_p + \Delta v_2)^2-\frac{2MG}{r_p}}$$

Where:

$v_p$: velocity at perhelion prior to the burn\\
$r_p$: perihelion distance\\
$\Delta v_2$: change in velocity from the second burn\\

The velocity and perihelion of \textit{Einstein} depends on the first burn. To solve for the perihelion:

$$r_p + r_0 = 2a$$
$$a = \bigg(\frac{2}{r_0} - \frac{(v_0-\Delta v_1)^2}{GM}\bigg)^{-1}$$

Where:
$r_p$: perihelion distance\\
$r_0$: starting distance from the sun - Jupiter's orbit\\
$\Delta v_1$: change in velocity of the first burn\\
$a$: semi-major axis\\

Note that $\Delta v_1$ subtracts from $v_0$ because the burn is opposite the direction of orbit to lower the perihelion. The second equation is a rearrangement of the equation for specific orbital energy. This equation gives us the perihelion distance. The velocity at the perihelion $v_p$ can be calculated from conservation of angular momentum:

$$r_p \times v_p = r_0 \times v_0$$

To summarize, $v_{\infty}$ depends on two factors: 

\begin{enumerate}
\item The burn at perihelion
\item How low the perihelion is, which in turn determined by the first burn. 
\end{enumerate}

Note that there is a limit to how low the perihelion can be - \textit{Einstein} can only fly so close to the sun. Figure 1 plots the hyperbolic excess velocity as a function of these two burns.

\begin{figure}[H]
\caption{This plot illustrates v-infinity as a function of the delta-v of the two burns. The first burn is performed while in a circular orbit around the sun near Jupiter and is retrograde, which lowers the perihelion. The second burn is performed at perihelion and puts \textit{Einstein} on an escape trajectory from the solar system. Each straight line represents a constant total delta-v.}
\end{figure}

\subsection{Comparing the two scenarios}

For lower delta-v budgets it is advantageous to perform only one burn out of the solar system. Figure 2 illustrates an example of v-infinity for a single burn and for two burns.

\begin{figure}[H]

\caption{V-infinity for the single burn and two burn scenarios. Performing two burns results in a lower v-infinity because the energy used to lower the orbit is not made up for by the Oberth effect.}
\end{figure}

However for larger delta-v budgets performing two burns is advantageous. Lowering the perihelion and burning at perihelion - and using the Oberth effect - can significantly increase v-infinity.

\begin{figure}[H]
\caption{V-infinity for the single burn and two burn scenarios where the total delta-v budget is X. Performing two burns significantly increases v-infinity}.
\end{figure}

In later sections we will see that the delta-v budget provided by the \textit{Sheridan} drive justifies performing two burns.


\section{Gravitational lensing focal distance}
One of the most challenging aspects of \textit{Einstein} is that it must travel nearly 600 AU (astronomical units - the average distance between the earth and sun) to exploit the gravitational lensing effect. Gravitational lensing was predicted by \textit{Einstein's} theory of general relativity. When light passes by a large object gravity warps its path. This is similar to how a comet will change trajectory as it passes near the sun. Light from a distant star - in the story Trappist 1 - is bent by the sun's gravity and concentrates at a focal point. Figure 5 below illustrates this.

\begin{figure}[H]

\caption{Illustration of gravitational lensing effect. Light passing by the sun is bent to a focal point. Light that passes further from the sun is bent to a further focal point, creating a focal line extending past the sun.}
\end{figure}

As light is bent around the sun it converges ar the focal point in a ring shaped pattern called an \textit{Einstein} ring. Figure X below is a real image of an \textit{Einstein} ring. In this case the object that bends starlight is a distant galaxy bending the light of an even more distant galaxy. 

\begin{figure}[H]
\makebox[\textwidth][c]{\includegraphics[width=5in]{einstein_ring.jpg}}
\caption{Image of an Einstein ring LRG 3-757. The light from a distant galaxy is bent into a circle by the gravity of a closer, redder galaxy (center).}
\end{figure}

The equation for the deflection angle caused by the lens is:

$$\theta = \frac{4GM}{c^2r}$$

where:

$\theta$ is the angle of deflection\\
$G$ is the universal gravitational constant\\
$M$ is the mass of the lensing object (the sun)\\
$c$ is the speed of light\\
$r$ is the closest approach light takes the sun\\

The derivation of the equation is beyond the scope of this appendix.

The focal distance $f_d$ can be written as:

$$f_d = r\tan (\frac{\pi}{2} - \theta)$$
$$f_d = r\tan\bigg(\frac{\pi}{2} - \frac{4GM}{c^2r}\bigg)$$

The minimum focal distance comes from light passing close to the sun's surface, which minimizes r. For $r$ as the radius of the sun ($6.96 \times 10^8 m$):

$$f_d =  548AU$$

However, light passing so close to the sun would be difficult to separate from the sun itself. In the story a combination of a starshade and additional filtering is used to block light from the sun. A starshade is a flat, roughly circular plate that separates from \textit{Einstein} and flies nearby to block sunlight. The further the light passing by the sun is - corresponding to a larger $r$ - the further the \textit{Einstein} ring will be from the sun and the easier it is to photograph and interpret. Figure X below illustrates the relation between $r$ and the focal distance. 

\begin{figure}[H]
	\makebox[\textwidth][c]{\includegraphics[width=5in]{image.png}}
	\caption{Focal distance as a function of how closely light from Trappist-1e passes to the sun. A greater focal distance takes longer to reach but leads to an \textit{Einstein} ring further from the sun that's easier to separate from the sun's light.}
\end{figure}

\textit{Einstein} must reach at least 548 AU to photograph an \textit{Einstein} ring of Trappist 1e. In practice, the telescope will need to travel than the minimum focal distance. As the telescope continues to coast through space and travel fruther from the sun the \text{Einstein} ring appears further from the sun. The total amount of light being bent into the ring also increases since the amount of light is proportional to the circumference of a circle around the sun with radius $r$. This means that the quality of images should improve as \textit{Einstein} travels further and further into space.

\section{Efficiency of \textit{Sheridan} drive}
\textit{Einstein} is powered by the \textit{Sheridan} drive, a closely guarded secret of the Callistan Navy. The drive is described as:

\hspace{2cm} \textit{a long sought after holy grail, a practical nuclear fusion engine}

This section will give some approximate specifications of the \textit{Sheridan} drive. Compared to other sections this section is lighter on detail and more speculative. Rather than attempt to sketch out the design of an actual nuclear fusion engine this section gives a general comparison between the \textit{Sheridan} drive and existing engines.

There are several choices of fuels for a nuclear fusion engine. The most likely are a deuterium tritium mix:
$$^2_1D^+ + ^3_1T^+ \rightarrow ^4_2He^{2+} (3.5 MeV) + n^0 (14.1 MeV)$$

or a helium-3 deuterium mix:

$$^2_1D^+ + ^3_2He^{2+} \rightarrow ^4_2He^{2+} (3.6 MeV) + p^+ (14.7 MeV)$$

Although the deuterium tritium reaction takes less energy to ignite, the helium-3 and deuterium mix is likely more favorable for a spacecraft engine. The byproducts of the second reaction are both charged and can be directed out of the engine bell by electromagnetic fields. The deuterium tritium reaction produces a neutral neutron which will be uniformly emitted in all directions. The neutron will also embed itself into the wall of the reactor, possibly embrittling and damaging the reactor.

The efficiency of an engine is related to its exhaust velocity. The exhaust velocity can be calculated from the kinetic energy:

$$E (joules) = \frac{1}{2}mv^2$$

Our energy units are in electron volts so the expression for velocity is:

$$v = \sqrt{\frac{2 E \frac{joules}{eV}}{m}}$$

for the helium-3 deuterim reaction this yields an exhaust velocity of:

$^4_2He^{2+}: 13,176 km/s$ or 0.044 c\\
$p^{+}: 53,059 km/s$ or 0.177 c\\

The exhuast of the \textit{Sheridan} drive is composed of two streams of particles with different exhaust velocities. The proton has an exhaust veloity of nearly 18\% the speed of light while the heavier alpha particle has a velocity of over 4\% the speed of light. The momentum change from the two exhaust streams is equivalent to the same mass being uniformly ejected at a velocity of:

$$p_{total} = 0.2mv_{p} + 0.8mv_{He} = mv_{combined}$$
$$v_{combined} = 0.2v_{p} + 0.8v_{He}$$
$$v_{combined} =  21,153km/s$$

which is equivalent to 7\% the speed of light. A common descriptor for the efficiency of an engine is its specific impulse, defined as the change in momentum per unit of fuel consumed.  When fuel is expressed by mass specific impulse is equivalent to exhaust velocity with units of meters per second. If weight is used then the units are expressed in seconds. The higher the exhaust velocity the greater the efficiency, and as we will see in the following section the faster a spacecraft can travel.

Since the \textit{Sheridan} drive is a first generation nuclear fusion engine we assume that its specific impulse is much less than the theoretical maximum. If we assume the specific impulse (and exhaust velocity) is an order of magnitude lower, then the specific impulse of the Sheridan drive is:

$$I_{sp} \approx 2,100 km/s \approx 214,000 seconds$$

Going forward we will round the \textit{Sheridan} drive's specific impulse to \textbf{200,000 seconds} and the exhaust velocity to \textbf{2,000} km/s.

The table below gives a comparison of the specific impulse of existing engines.

\begin{center}
\begin{tabular}{|m{5 cm}| m{5 cm}|} \hline
\textbf{Engine} & \textbf{Specific impulse (s)}\\ \hline
Falcon 9 Merlin-1D &  311 \\ \hline
Space shuttle main engine &  453 \\ \hline
Dawn ion engine   &  3100\\ \hline
\textit{Sheridan} drive & 200,000\\ \hline
\end{tabular}
\end{center}

The specific impulse of the \textit{Sheridan} drive is nearly two orders of magnitude greater than that of the \textit{Dawn} ion engines. We will see in the next section that the specific impulse is a strong driver of $\Delta v$, which in turn determines how long it takes \textit{Einstein} to reach the sun's focal point.

\section{\textit{Einstein} flight time}
In the story \textit{Einstein} takes ``decades" to reach the sun's focal point. This section calculates how long the trip takes based on the specific impulse of the \textit{Sheridan} drive and the distance to the focal point. Section 4 is more technically challenging than the first three sections and includes descriptions of several numerical approximations used to calculate \textit{Einstein}'s flight time.

The $\Delta v$ budget of a spacecraft is determined by the Tsiolkovsky rocket equation:

$$\Delta v = v_e ln \frac{m_0}{m_f}$$

where:

$v_e$ is exhaust velocity\\
$m_0$ is the starting mass\\
$m_f$ is the ending mass after all fuel is expended\\

The $\Delta v$ strongly depends on the fraction of the spacecraft taken up by fuel. For \textit{Einstein} we assume that the exhaust velocity is 2,000 km/s and that the fraction of the starting mass taken up by fuel is 10\%. For comparison, NASA's \textit{Cassini-Huygens} probe was more than 50\% fuel at launch. Since \textit{Einstein} is a first generation nuclear fusion engine with a heavy reactor we assume it has a lower fuel ratio. Substituting into the rocket equation gives a $\Delta v$ of:

$$\Delta v = 2,000,000\frac{m}{s} \times ln \frac{1}{0.9}$$
$$\Delta v \approx 210 km/s$$ 

The flight to the focal point can be broken into several phase:
\begin{enumerate}
\item Burn to Jupiter escape
\item Coast to perihelion
\item Burn at perihelion
\item Coast to focal point
\end{enumerate}

\subsection{Burn to Jupiter escape}
The first leg of the trip is to escape from Jupiter's gravity well. \textit{Einstein} is launched from Callisto and flies on an elliptical orbit that takes it near Jupiter. When the craft is near Jupiter the \textit{Sheridan} drive performs its first burn to escape from Jupiter's gravity well and put it on a trajectory close to the sun.

The transit time from Callisto to Jupiter is on the order of several days to a few weeks. Leaving Jupiter's gravity well takes a similar amount of time. The time needed to escape from Jupiter will be ignored since it's relatively small compared to the flight time to the focal point.

In section 1 of the appendix we calculated that a $\Delta v$ of about 10 km/s is needed to put \textit{Einstein} on a trajectory taking it close to the sun. The actual figure should be smaller since the burn to escape Jupiter and enter a sun diving orbit takes place near Jupiter, where the Oberth effect makes a burn more effective. For simplicity we will use the 10 km/s figure.

$\Delta$ \textbf{v:} 10 km/s\\
\textbf{Time:} 0

\subsection{Coast to perihelion}
The second leg of the trip is when \textit{Einstein} coasts from Jupiter inwards towards the sun. The spacecraft follows an elliptical orbit and does not expend any fuel. The period of an orbit is given by Kepler's second law:

$$T^2 \propto a^3$$

which states that the period of an orbit squared is proportional to the semi major axis cubed. The earth has a period of one year and a semi-major axis of almost exactly one AU. Using the units of years and AU the ratio between period squared and semi-major axis cubed is one.

The elliptical orbit that \textit{Einstein} takes has a semi-major axis of:

$$a = \frac{1}{2}(r_{perihelion} + r_{aphelion})$$

where
$r_{perihelion}$ = closest approach to the sun = 0.1 AU\\
$r_{aphelion}$ = further distance to the sun = 5 AU\\

The perihelion is limited by how close \textit{Einstein} can be to the sun without overheating. For comparison NASA's Parker Solar Probe, which has specially designed thermal protection, will fly within 0.046 AU of the sun. Substituting into Kepler's law gives a period of:

$$1 = \frac{T^2}{a^3}$$
$$2.05AU^{3} = T^2$$
$$T = 4.1 years$$

The period of \textit{Einstein}'s elliptical trajectory is about 4 years. The time it takes \textit{Einstein} to complete half an orbit (starting from the aphelion and ending at perihelion) is half that, or about 2 years.

$\Delta$ \textbf{v:} 0 km/s\\
\textbf{Time:} 2 years

\subsection{Burn at perihelion}
The burn at perihelion is the \textit{Sheridan} drive's longest and most powerful burn. In the story the engine burns continuously for several days, accelerating \textit{Einstein} out of the solar system. In this section we'll calculate \textit{Einstein}'s v-infinity after performing the burn. We're going to make two approximations that will make calculations much easier while giving us a reasonable approximation:

\begin{enumerate}
\item \textit{Einstein}'s trajectory is a straight line that starts at the perihelion and continues perpendicular to a line connecting the perihelion to the sun.
\item The long, multi day burn is approximated as a series of discrete, instantaneous velocity changes separated by periods when \textit{Einstein} coasts. During the coast phases \textit{Einstein}'s velocity is governed only by the effect of gravity.
\end{enumerate}

In reality, \textit{Einstein} follows a hyperbolic path that constantly changes as the burn continues. The approximation underestimates the final v-infinity because the straight path moves away from the sun more quickly than a hyperbolic path that curves slightly inwards towards the sun. Leaving the sun's gravity well more quickly reduces the Oberth effect, and leads to a lower v-infinity. However, it is still a \textit{fairly} good approximation. Figure X below illustrates how the approximation compares to a more realistic calculation.

\begin{figure}[H]

\caption{Comparison of \textit{Einstein}'s actual velocity and the approximated trajectory.}

\end{figure}


Even with these approximations, it is difficult to directly solve for \textit{Einstein}'s final velocity. Instead, we will use the following steps to approximate the telescope's velocity after it's perihelion burn:

\begin{enumerate}
\item Calculate the average acceleration of the entire burn
\item Instantaneously increase the velocity a small amount
\item Use a linear approximation to estimate how far \textit{Einstein} should coast for so that the magnitude of the velocity change in step 2 divided by the coast time is equal to the average acceleration.
\item Use Simpson's rule to calculate how much time elapses when \textit{Einstein} coasts for the estimated distance
\item Update \textit{Einstein}'s position and velocity after the coast phase as it cruises on a straight line.
\item Repeat steps 2-5 until the burn and the total velocity change is complete. This breaks the continuous, multi-day burn into many small burns.
\end{enumerate}

The first two steps are straightforward - the total $\Delta v$ is divided by the number of steps that we breaking the burn into. The result is the instantaneous velocity change at each step. 

\subsubsection{Step 3: Estimating coast time}
The goal of steps 3-5 is to find what \textit{Einstein}'s new position and velocity are at the end of the cruise phase. After the each small burn (step 2), we let \textit{Einstein} cruise for a period of time. The time that it cruises between burns should give an average acceleration equal to what we calculated in step 1. 

The relation between cruise time and cruise distance is given by: 

$$T = \int_{r_0}^{r_f} \frac{1}{v(r)} dr$$

If we can solve for this integral analytically, then we can rearrange it and get an explicit formula for the cruise distance as a function of time. Updating \textit{Einstein}'s position is then simply plugging in the desired cruise time between burns and calculating its new position. Unfortunately, it's more complicated.

The equation for velocity as a function of distance from the sun $r$ comes from the equation for specific orbital energy. Since the telescope is coasting, the sum of its kinetic and potential energy is constant:

$$C_3 = \frac{1}{2}v_0^2-\frac{M_sG}{r_0}$$
$$C_3 = \frac{1}{2}(v_r)^2 - \frac{M_sG}{r_0}$$
$$v(r) = \sqrt{2  (C_3 + \frac{M_sG}{r})}$$
$$v(r) = \sqrt{2  (\frac{1}{2}v_0^2-\frac{M_sG}{r_0} + \frac{M_sG}{r})}$$
$$v(r) = \sqrt{v_0^2-\frac{2M_sG}{r_0} + \frac{2M_sG}{r})}$$

Substituting this equation into the equation for time gives an integral that cannot be solved analytically. Instead, we can use a linear approximation for the velocity as a function of distance. The linear form of the time integral is:

$$\boxed{T = \int_{r_0}^{r_f} \frac{1}{v_0+v'(r_0)r} dr}$$

where:\\
$v_0$= velocity at the beginning of the cruise phase

using $m$ for the derivative of velocity at $r_0$, we can solve for the integral using $u$ substitution:

$$T = \int_{r_0}^{r_f} \frac{1}{v_0+mr} dr$$
$$u = v_0+mr\hspace{10mm}du = mdr\hspace{10mm}dr = \frac{1}{m}du$$
$$T = \int \frac{1}{u} \frac{1}{m}dr$$
$$T = \frac{1}{m} \log{u}$$
$$T = \frac{1}{m} \log{(v_0+mr)}\bigg]_{r_0}^{r_f}$$
$$\boxed{T = \frac{1}{m} \bigg(\log{(v_0+mr_f)} - \log{(v_0+mr_0)}\bigg)}$$

We want an expression for the final position for a given coast time so we rearrange the equation to get:

$$mT = \log{(v_0+mr_f)} - \log{(v_0+mr_0)}$$

$$\log{(v_0+mr_f)} = \log{(v_0+mr_0)} + mT$$
$$v_0 + mr_f = e^{\log{(v_0+mr_0)}{mT}}$$
$$v_0 + mr_f = e^{mT}(v_0+mr_0)$$
$$\boxed{r_f = \frac{1}{m}\bigg(e^{mT}(v_0+mr_0) - v_0\bigg)}$$


The only piece of this equation we're missing is an expression for $m$, the slope of the velocity as a function of distance traveled $d$. We can solve for this by taking the derivative of $v(r)$ and applying the chain rule:

$$\frac{dv}{dd} = \frac{dv}{dr}\times\frac{dr}{dd}$$

Solving for $\frac{dv}{dr}$:

$$\frac{dv}{dr} = \frac{1}{2} \bigg(v_0^2-\frac{2M_sG}{r_0} + \frac{2M_sG}{r}\bigg)^{-\frac{1}{2}}\frac{dv}{dr}\bigg(v_0^2-\frac{2M_sG}{r_0} + \frac{2M_sG}{r}\bigg)$$

$$\frac{dv}{dr} = \frac{1}{2} \bigg(v_0^2-\frac{2M_sG}{r_0} + \frac{2M_sG}{r}\bigg)^{-\frac{1}{2}}2M_sG\log{r}$$

$$\frac{dv}{dr} = \frac{2M_sG \log{r}}{2 \sqrt{v_0^2-\frac{2M_sG}{r_0} + \frac{2M_sG}{r}}}$$

The expression for $\frac{dv}{dr}$ is complicated and we still need to use the chain rule. Instead of explicitly calculating the derivative, we can approximate the derivative by calculating the velocity at two points and solving for the slope:

$$\frac{dv}{dd} \approx \frac{v(d_0 +\Delta) - v(d_0)}{\Delta}$$


$\Delta$ \textbf{v:} 200 km/s\\
\textbf{Time:} 10 days\\

\subsection{Coast to focal point}
After its final burn \textit{Einstein} coasts to the focal point. As the spacecraft flies further from the sun it slows down, exchanging kinetic energy for potential energy. The velocity of \textit{Einstein} as a function of its distance from the sun $r$ is equivalent to its total energy at perihelion:

$$E_{p} = \frac{1}{2}v_p^2 - \frac{MG}{r_p} = \frac{1}{2}v^2-\frac{MG}{r}$$
$$\frac{1}{2}v^2 = E_p+\frac{MG}{r}$$
$$v = \sqrt{2\big(E_p + \frac{MG}{r}\big)}$$

Figure X plots the velocity of \textit{Einstein} as a function of distance from the sun.

To calculate \textit{Einstein}'s flight time we integrate one over the velocity:

$$T = \int_{r_0}^{r_f} \frac{1}{v(r)} dr$$


\end{document}
