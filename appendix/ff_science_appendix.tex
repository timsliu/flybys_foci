\documentclass[12pt]{article} % use larger type; default would be 10pt

%packages
\usepackage[utf8]{inputenc} % set input encoding (not needed with XeLaTeX)
\usepackage{fancyhdr}
\usepackage{float}
\usepackage{geometry}
\usepackage{ulem}
\usepackage{soul}
\usepackage{color}
\usepackage{graphicx}
\usepackage{hyperref}
\usepackage{array}
\usepackage{caption}
\usepackage{titling}
\usepackage{enumerate} 
\usepackage[dvipsnames]{xcolor}
\usepackage{amsmath}
\usepackage{amssymb}
\usepackage[compact]{titlesec}


 %put box around figure captions
\makeatletter
\long\def\@makecaption#1#2{%
  \vskip\abovecaptionskip
  \sbox\@tempboxa{\fbox{#1: #2}}%
  \ifdim \wd\@tempboxa >\hsize
    \fbox{\parbox{\dimexpr\linewidth-2\fboxsep-2\fboxrule}{#1: #2}}\par
  \else
    \global \@minipagefalse
    \hb@xt@\hsize{\hfil\box\@tempboxa\hfil}%
  \fi
  \vskip\belowcaptionskip}
\makeatother

%reduce space between sections
\titlespacing{\section}{0pt}{*1}{*0}
\titlespacing{\subsection}{0pt}{*1}{*0}
\titlespacing{\subsubsection}{0pt}{*0}{*0}


%no indent and modify distance between paragraphs
\setlength\parindent{0pt}
\setlength\parskip{12pt}

%set margins and line spacing
\geometry{margin=1in}
\linespread{1.2}
\geometry{letterpaper}

%math operators
\DeclareMathOperator{\E}{\mathbb{E}}

%set up header and page numbering
\pagestyle{fancy}
\lhead{Scientific Appendix}
\rhead{Timothy Liu}
\pagenumbering{arabic}



\title{Flybys and Foci Scientific Appendix}
\author{Timothy Liu}

\begin{document}

\maketitle

Throughout the story I have done my best to keep the plot scientifically accurate and plausible. This section gives a more thorough explanation of some parts of the story that the reader may find interesting. The appendix is written assuming the reader has some basic knowledge of physics, to the level that can be found on Wikipedia.

\section{Appendix A: Diving Towards the Sun}


In Chapter 8, \textit{Einstein} is put on a trajectory that passes close to the sun before performing its main burn to exit the solar system. This may seem unintuitive - the spacecraft first heads towards the sun to make it to a point very distant to the sun. \textit{Einstein} is taking advantage of the Oberth Effect, where the most efficient place to perform an engine burn is deep in a gravity well.

The total energy of a body in orbit is the sum of the kinetic and potential energy:

$$ E = \frac{1}{2} mv^2 - \frac{mMG}{r} = -\frac{mMG}{2a}$$

Where:

$m$ = spacecraft mass \\
$v$ = velocity \\
$M$ = solar mass\\
$G$ = universal gravitational constant\\
$r$ = distance to the sun\\
$a$ = semi-major axis\\

This expression is commonly divided by mass to get the \textit{specific orbital energy}:

$$ = \frac{1}{2} v^2 - \frac{MG}{r} = -\frac{MG}{2a}$$

For \textit{Einstein} the higher the specific orbital energy the faster it will reach the suns focal point. A more tangible measure is $v_{\infty}$, which is the spacecraft's velocity when it is so far away from the sun that its potential energy is negligible. As a spacecraft climbs out of the sun's gravity well it sheds velocity and the potential energy goes to 0 (in orbital mechanics potential energy is 0 at infinity and increasingly negative closer to the sun).

$$\frac{1}{2}v_{\infty}^{2} = \frac{1}{2} v^2 - \frac{MG}{r} = -\frac{MG}{2a}$$
$$v_{\infty} = \sqrt{v^2-\frac{2MG}{r}}$$

Where:

$r$ = spacecraft distance to the sun\\
$v$ = velocity at the given $r$\\

We can use this to calculate $v_{\infty}$ for \textit{Einstein} if it had burned directly on a hyperbolic escape trajectory and compare it to $v_{\infty}$ from its trajectory that took it close to the sun.

We can use this to calculate $v_{\infty}$ for \textit{Einstein} if it had burned directly on a hyperbolic escape trajectory and compare it to $v_{\infty}$ from its actual trajectory that took it close to the sun.

\subsection{Scenario 1: Direct escape burn}

Assume that \textit{Einstein} begins in a heliocentric, circular orbit the same distance from the sun as Jupiter. In the story \textbf{Einstein} must first escape from Jupiter's gravity well but for simplicity we'll assume this doesn't require much $\Delta$ v.

$$\frac{1}{2}v_{\infty}^{2} = \frac{1}{2} (v_0 + \Delta v)^2 - \frac{MG}{r_0}$$
$$v_{\infty} = \sqrt{(v_0 + \Delta v)^2-\frac{2MG}{r_0}}$$

Where:
$v_0$: starting velocity\\
$\Delta v$: change in velocity supplied by \textit{Sheridan} drive\\
$r_0$: starting distance from the sun - Jupiter's orbit


\subsection{Scenario 2: Oberth effect}

Again we assume that \textit{Einstein} begins in a heliocentric, circular orbit the same distance from the sun as Jupiter. \textit{Einstein} then performs two burns - one retrograde (opposite the direction of orbit) to bring down the perihelion (the point in the orbit nearest the sun) and a second burn near the sun to escape the solar system. In the story the first burn is combined with the maneuver to escape Jupiter, again is taking advantage of the Oberth effect!

The hyperbolic excess velocity ($\Delta v_{\infty}$) following the second burn is:

$$v_{\infty} = \sqrt{(v_p + \Delta v_2)^2-\frac{2MG}{r_p}}$$

Where:

$v_p$: velocity at perhelion prior to the burn\\
$r_p$: perihelion distance\\
$\Delta v_2$: change in velocity from the second burn\\

The velocity and perihelion of \textit{Einstein} depends on the first burn. To solve for the perihelion:

$$r_p + r_0 = 2a$$
$$a = \bigg(\frac{2}{r_0} - \frac{(v_0-\Delta v_1)^2}{GM}\bigg)^{-1}$$

Where:
$r_p$: perihelion distance\\
$r_0$: starting distance from the sun - Jupiter's orbit\\
$\Delta v_1$: change in velocity of the first burn\\
$a$: semi-major axis\\

Note that $\Delta v_1$ subtracts from $v_0$ because the burn is opposite the direction of orbit to lower the perihelion. The second equation is a rearrangement of the equation for specific orbital energy. This equation gives us the perihelion distance. The velocity at the perihelion $v_p$ can be calculated from conservation of angular momentum:

$$r_p \times v_p = r_0 \times v_0$$

To summarize, $v_{\infty}$ depends on two factors: 

\begin{enumerate}
\item The burn at perihelion
\item How low the perihelion is, which in turn determined by the first burn. 
\end{enumerate}

Note that there is a limit to how low the perihelion can be - \textit{Einstein} can only fly so close to the sun. Figure 1 plots the hyperbolic excess velocity as a function of these two burns.

\begin{figure}
\caption{This plot illustrates v-infinity as a function of the delta-v of the two burns. The first burn is performed while in a circular orbit around the sun near Jupiter and is retrograde, which lowers the perihelion. The second burn is performed at perihelion and puts \textit{Einstein} on an escape trajectory from the solar system. Each straight line represents a constant total delta-v.}
\end{figure}

\subsection{Comparing the two scenarios}

For lower delta-v budgets it is advantageous to perform only one burn out of the solar system. Figure 2 illustrates an example of v-infinity for a single burn and for two burns.

\begin{figure}

\caption{V-infinity for the single burn and two burn scenarios. Performing two burns results in a lower v-infinity because the energy used to lower the orbit is not made up for by the Oberth effect.}
\end{figure}

However for larger delta-v budgets performing two burns is advantageous. Lowering the perihelion and burning at perihelion - and using the Oberth effect - can significantly increase v-infinity.

\begin{figure}
\caption{V-infinity for the single burn and two burn scenarios where the total delta-v budget is X. Performing two burns significantly increases v-infinity}.
\end{figure}

In later sections we will see that the delta-v budget provided by the \textit{Sheridan} drive justifies performing two burns.


\section{Gravitational lensing focal distance}
One of the most challenging aspects of \textit{Einstein} is that it must travel nearly 600 AU (astronomical units - the average distance between the earth and sun) to exploit the gravitational lensing effect. Gravitational lensing was predicted by \textit{Einstein's} theory of general relativity. When light passes by a large object gravity warps its path. This is similar to how a comet will change trajectory as it passes near the sun. Light from a distant star - in the story Trappist 1 - is bent by the sun's gravity and concentrates at a focal point. Figure 5 below illustrates this.

\begin{figure}
\caption{Illustration of gravitational lensing effect. Light passing by the sun is bent to a focal point. Light that passes further from the sun is bent to a further focal point, creating a focal line extending past the sun.}
\end{figure}

As light is bent around the sun it converges ar the focal point in a ring shaped pattern called an \textit{Einstein} ring. Figure X below is a real image of an \textit{Einstein} ring. In this case the object that bends starlight is a distant galaxy bending the light of an even more distant galaxy. 

\begin{figure}

\end{figure}

The equation for the deflection angle caused by the lens is:

$$\theta = \frac{4GM}{c^2r}$$

where:

$\theta$ is the angle of deflection\\
$G$ is the universal gravitational constant\\
$M$ is the mass of the lensing object (the sun)\\
$c$ is the speed of light\\
$r$ is the closest approach light takes the sun\\

The derivation of the equation is beyond the scope of this appendix.

The focal distance $f_d$ can be written as:

$$f_d = r\tan \theta$$
$$f_d = r\tan{\frac{4GM}{c^2r}}$$

The minimum focal distance comes from light passing close to the sun's surface, which minimizes r. For $r$ as the radius of the sun ($6.96 \times 10^8 m):

$f_d =  = AU$

However, light passing so close to the sun would be difficult to separate from the sun itself. In the story a combination of a starshade and additional filtering is used to block light from the sun. A starshade is a flat, roughly circular plate that separates from \textit{Einstein} and flies nearby to block sunlight. The further the light passing by the sun is - corresponding to a larger $r$ - the further the \textit{Einstein} ring will be from the sun and the easier it is to photograph and interpret. Figure X below illustrates the relation between $r$ and the focal distance. 

\begin{figure}

\end{figure}

\textit{Einstein} must reach at least xxx AU to photograph an \textit{Einstein} ring of Trappist 1e. As the telescope continues to coast through space and travel fruther from the sun the \text{Einstein} ring appears further from the sun. The total amount of light being bent into the ring also increases since the amount of light is proportional to the circumference of a circle around the sun with radius $r$. This means that the quality of images should improve as \textit{Einstein} travels further and further into space.

\section{Efficiency of \textit{Sheridan} drive}

\section{\textit{Einstein} flight time}

\end{document}
