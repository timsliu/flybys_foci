\documentclass[12pt]{article} % use larger type; default would be 10pt

%packages
\usepackage[utf8]{inputenc} % set input encoding (not needed with XeLaTeX)
\usepackage{fancyhdr}
\usepackage{float}
\usepackage{geometry}
\usepackage{ulem}
\usepackage{soul}
\usepackage{color}
\usepackage{graphicx}
\usepackage{hyperref}
\usepackage{array}
\usepackage{caption}
\usepackage{titling}
\usepackage{enumerate} 
\usepackage[dvipsnames]{xcolor}
\usepackage{amsmath}
\usepackage{amssymb}
\usepackage[compact]{titlesec}


 %put box around figure captions
\makeatletter
\long\def\@makecaption#1#2{%
  \vskip\abovecaptionskip
  \sbox\@tempboxa{\fbox{#1: #2}}%
  \ifdim \wd\@tempboxa >\hsize
    \fbox{\parbox{\dimexpr\linewidth-2\fboxsep-2\fboxrule}{#1: #2}}\par
  \else
    \global \@minipagefalse
    \hb@xt@\hsize{\hfil\box\@tempboxa\hfil}%
  \fi
  \vskip\belowcaptionskip}
\makeatother

%reduce space between sections
\titlespacing{\section}{0pt}{*1}{*0}
\titlespacing{\subsection}{0pt}{*1}{*0}
\titlespacing{\subsubsection}{0pt}{*0}{*0}


%no indent and modify distance between paragraphs
\setlength\parindent{0pt}
\setlength\parskip{12pt}

%set margins and line spacing
\geometry{margin=1in}
\linespread{1.2}
\geometry{letterpaper}

%math operators
\DeclareMathOperator{\E}{\mathbb{E}}

%set up header and page numbering
\pagestyle{fancy}
\lhead{Scientific Appendix}
\rhead{Timothy Liu}
\pagenumbering{arabic}

\hypersetup{  %set up url
    colorlinks=true,
    linkcolor=blue,
    filecolor=magenta,      
    urlcolor=cyan,
}



\title{Flybys and Foci Scientific Appendix}
\author{Timothy Liu}

\begin{document}

\maketitle


% Scientific appendix
Throughout the story I have done my best to keep the plot scientifically accurate and plausible. This appendix gives a more thorough explanation of some parts of the story that the reader may find interesting. I have written the appendix assuming the reader has some basic knowledge of physics, and much of the background information to understand this section can be found on Wikipedia.

The appendix is organized into four sections. \hyperref[sec:diving]{Section 1} explains the flight path through space \textit{Einstein} takes and why it makes a flyby of the sun first before heading to deep space. The \hyperref[sec:lensing]{second section} details the gravitational lensing effect and calculates how far away the sun's focal point is. \hyperref[sec:efficiency]{Section 3} gives a rough estimate of how efficient the \textit{Sheridan} drive is and how it stacks up against current engines. Finally, \hyperref[sec:flighttime]{Section 4} is the culmination of the first three sections and calculates how long the flight to the sun's focal point is.

\newpage
\tableofcontents
\newpage

% section - diving towards the sun
\section{Diving Towards the Sun}
\label{sec:diving}

In Chapter 11, \textit{Einstein} is put on a trajectory that passes close to the sun before performing its main burn to exit the solar system. This may seem counterintuitive - the spacecraft first heads towards the sun in the first leg of a journey away from the sun. \textit{Einstein} is taking advantage of the Oberth Effect, where the most efficient place to perform an engine burn is deep in a gravity well.

In rocketry a common unit for describing the effort needed to travel between points in space is delta v, also expressed as $\Delta v$. Delta v is the change in velocity needed to modify a spacecraft's orbit. For example, to go from low earth orbit to a trajectory that escapes from earth's gravity takes a change in velocity (or delta v) of 3.2 km/s. The total velocity change that a spacecraft can perform is called the delta v budget. This section explains how \textit{Einstein}'s trajectory in the story maximizes its delta v budget to reach the focal point as quickly as possible.

The total energy of a body in orbit is the sum of the kinetic and potential energy:

$$ E = \frac{1}{2} mv^2 - \frac{mMG}{r} = -\frac{mMG}{2a}$$

Where:

$m$ = spacecraft mass \\
$v$ = velocity \\
$M$ = mass of body being orbited\\
$G$ = universal gravitational constant\\
$r$ = distance to the sun\\
$a$ = semi-major axis\\

This expression is commonly divided by mass to get the \textit{specific orbital energy}:

$$ = \frac{1}{2} v^2 - \frac{MG}{r} = -\frac{MG}{2a}$$

For \textit{Einstein}, the higher its specific orbital energy the faster it will reach the sun's focal point. A more direct measure of speed is $v_{\infty}$, which is the spacecraft's velocity when it is so far away from the sun that its potential energy is negligible. As a spacecraft climbs out of the sun's gravity well it sheds velocity and the potential energy goes to 0 (in orbital mechanics potential energy is 0 at infinity and increasingly negative closer to the sun).

$$\frac{1}{2}v_{\infty}^{2} = \frac{1}{2} v^2 - \frac{MG}{r} = -\frac{MG}{2a}$$
$$v_{\infty} = \sqrt{v^2-\frac{2MG}{r}}$$

Where:

$r$ = spacecraft distance to the sun\\
$v$ = velocity at the given $r$\\

We can use this to calculate $v_{\infty}$ for \textit{Einstein} if it had burned directly on a hyperbolic escape trajectory (straight out of the solar system away from the sun) and compare it to $v_{\infty}$ from its trajectory that took it close to the sun. A higher $v_{\infty}$ corresponds to a faster path and a shorter flight time to the focal point.

% Subsection - direct escape
\subsection{Scenario 1: Direct escape burn (single burn)}

Assume that \textit{Einstein} begins in a heliocentric (sun centered), circular orbit the same distance from the sun as Jupiter. In the story \textit{Einstein} must first escape from Jupiter's gravity well but for simplicity we'll assume this doesn't require much $\Delta v$. The expression for $v_{\infty}$ is:

$$\frac{1}{2}v_{\infty}^{2} = \frac{1}{2} (v_0 + \Delta v)^2 - \frac{MG}{r_0}$$
$$v_{\infty} = \sqrt{(v_0 + \Delta v)^2-\frac{2MG}{r_0}}$$

Where:\\

$v_0$: starting velocity\\
$\Delta v$: change in velocity supplied by \textit{Sheridan} drive\\
$r_0$: starting distance from the sun - Jupiter's orbit

% subsection - oberth
\subsection{Scenario 2: Oberth effect (two burns)}

Again we assume that \textit{Einstein} begins in a heliocentric, circular orbit the same distance from the sun as Jupiter. \textit{Einstein} then performs two burns - one retrograde (opposite the direction of orbit) to bring down the perihelion (the point in the orbit nearest the sun) and a second burn near the sun to escape the solar system. In the story this first burn is combined with the maneuver to escape Jupiter, again taking advantage of the Oberth effect!

The hyperbolic excess velocity ($ v_{\infty}$) following the second burn is:

$$v_{\infty} = \sqrt{(v_p + \Delta v_2)^2-\frac{2MG}{r_p}}$$

Where:

$v_p$: velocity at perhelion prior to the second burn\\
$r_p$: perihelion distance\\
$\Delta v_2$: change in velocity from the second burn\\

The velocity and perihelion of \textit{Einstein} depends on the first burn. To solve for the perihelion as a function of the first burn $r_p$ we use two formulas:

$$r_p + r_0 = 2a$$
$$a = \bigg(\frac{2}{r_0} - \frac{(v_0-\Delta v_1)^2}{GM}\bigg)^{-1}$$

Where:
$r_p$: perihelion distance\\
$r_0$: starting distance from the sun - Jupiter's orbit\\
$\Delta v_1$: change in velocity of the first burn\\
$a$: semi-major axis\\

Note that $\Delta v_1$ subtracts from $v_0$ because the burn is opposite the direction of orbit to lower the perihelion. The second equation is a rearrangement of the equation for specific orbital energy. This equation gives us the perihelion distance. The velocity at the perihelion $v_p$ can then be calculated from conservation of angular momentum:

$$r_p \times v_p = r_0 \times v_0$$

To summarize, $v_{\infty}$ depends on two factors: 

\begin{enumerate}
\item The burn at perihelion
\item How low the perihelion is, which in turn determined by the first burn. 
\end{enumerate}

Note that there is a limit to how low the perihelion can be - \textit{Einstein} can only fly so close to the sun. Figure~\ref{fig:TwoBurns} plots the hyperbolic excess velocity as a function of these two burns.

\begin{figure}[H]
\makebox[\textwidth][c]{\includegraphics[width=5in]{v_infinity.png}}
\caption{This plot illustrates $v_{\infty}$ as a function of the $\Delta v$ of the two burns. The first burn is performed while in a circular orbit around the sun near Jupiter and is retrograde, which lowers the perihelion. The second burn is performed at perihelion and puts \textit{Einstein} on an escape trajectory from the solar system. Each straight line represents a constant total $\Delta v$. The greater the first burn (and the lower the perihelion) the higher the final $v_{\infty}$.}
\label{fig:TwoBurns}
\end{figure}

% comparing scenarios
\subsection{Comparing the two scenarios}

For lower $\Delta v$ budgets, it is advantageous to perform only one burn directly out of the solar system. Figure~\ref{fig:lowdv} compares the $v_{\infty}$ for the single burn and the two burn scenario for a smaller $\Delta v$ budget of 12km/s.

\begin{figure}[H]
\makebox[\textwidth][c]{\includegraphics[width=5in]{compare_12_kms.png}}
\caption{$v_{\infty}$ for the single burn and two burn scenarios with a $\Delta v$ budget of 12km/s. Performing two burns results in a lower $v_{\infty}$ because the energy used to lower the orbit is not made up for by the Oberth effect. After the first burn, there is not enough $\Delta v$ left to achieve a higher $v_{\infty}$.}
\label{fig:lowdv}
\end{figure}

However for larger $\Delta v$ budgets two burn scenario is strongly advantageous. Lowering the perihelion and burning at perihelion - and using the Oberth effect - can significantly increase $v_{\infty}$ compared to the single burn scenario if the $\Delta v$ budget is large enough.

\begin{figure}[H]
\makebox[\textwidth][c]{\includegraphics[width=5in]{compare_50_kms.png}}
\caption{$v_{\infty}$ for the single burn and two burn scenarios where the total $\Delta v$ budget is 50km/s. Performing two burns significantly increases $v_{\infty}$, and the gains increase when more $\Delta v$ is spent on the first burn.}
\label{fig:highdv}
\end{figure}

Above a certain $\Delta v$ budget threshold, it's advantageous to first dive in close to the sun and then burn at perihelion. In later sections we will see that the $\Delta v$ budget provided by the \textit{Sheridan} drive justifies performing two burns.

% section gravitational lensing
\section{Gravitational lensing focal distance}
\label{sec:lensing}
One of the most challenging aspects of the \textit{Einstein} project is that the telescope must travel nearly 600 AU (astronomical units - the average distance between the earth and sun) to exploit the gravitational lensing effect. Gravitational lensing was predicted by Einstein's theory of general relativity. When light passes by a large object, gravity warps its path. This is similar to how a comet will change trajectory as it passes near the sun. Light from a distant star - in the story Trappist 1e - is bent by the sun's gravity and concentrates at a focal point. Figure ~\ref{fig:lensing} below illustrates this effect.

\begin{figure}[H]
\makebox[\textwidth][c]{\includegraphics[width=5in]{lensing.png}}
\caption{Illustration of gravitational lensing effect. Light passing by the sun is bent to a focal point. Light that passes further from the sun is bent to a further focal point, creating a focal line extending past the sun.}
\label{fig:lensing}
\end{figure}

As light is bent around the sun it converges at the focal point in a ring shaped pattern called an Einstein ring. Figure~\ref{fig:reallense} below is an image of an Einstein ring taken by the Hubble Space Telescope. In this case the object that bends starlight is a distant galaxy bending the light of an even more distant galaxy. 

\begin{figure}[H]
\makebox[\textwidth][c]{\includegraphics[width=4in]{horseshoe.jpeg}}
\caption{Image of an Einstein ring LRG 3-757. The light from a distant galaxy is bent into a circle by the gravity of a closer, redder galaxy (center).\\ \url{https://en.wikipedia.org/wiki/Cosmic_Horseshoe}}
\label{fig:reallense}
\end{figure}

The equation for the deflection angle of light caused by a massive object is:

$$\theta = \frac{4GM}{c^2r}$$

where:

$\theta$ is the angle of deflection relative to a straight path\\
$G$ is the universal gravitational constant\\
$M$ is the mass of the lensing object (the sun)\\
$c$ is the speed of light\\
$r$ is the closest approach light takes the sun\\

The derivation of the equation is beyond the scope of this appendix.

The focal distance $f_d$ can be written as:

$$f_d = r\tan (\frac{\pi}{2} - \theta)$$
$$f_d = r\tan\bigg(\frac{\pi}{2} - \frac{4GM}{c^2r}\bigg)$$

The minimum focal distance comes from light passing close to the sun's surface, which minimizes r. For $r$ as the radius of the sun ($6.96 \times 10^8 m$):

$$f_d =  548AU$$

However, at this distance the \textit{Einstein} ring would appear right on the surface of the sun and would be difficult to separate from the sun itself. In the story a combination of a starshade and additional filtering is used to block light from the sun. A starshade is a flat, roughly circular plate that separates from \textit{Einstein} and flies nearby to block sunlight. The further the light passing by the sun is - corresponding to a larger $r$ - the further the \textit{Einstein} ring will be from the sun and the easier it is to photograph and interpret. Figure~\ref{fig:lensingdistance} below illustrates the relation between $r$ and the focal distance. 

\begin{figure}[H]
	\makebox[\textwidth][c]{\includegraphics[width=5in]{lens_distance.png}}
	\caption{Focal distance as a function of how closely light from Trappist-1e passes to the sun. A greater focal distance takes longer to reach but leads to an \textit{Einstein} ring further from the sun that's easier to separate from the sun's light.}
	\label{lensingdistance}
\end{figure}

As the telescope continues to coast through space and travel further along the focal line, the \text{Einstein} ring appears further from the sun. The total amount of light being bent into the ring also increases since the amount of light is proportional to the circumference of a circle around the sun with radius $r$. As a result, the quality of images should improve as \textit{Einstein} travels further and further along the focal line.

% section - Sheridan drive efficiency
\newpage
\section{Efficiency of \textit{Sheridan} drive}
\label{sec:efficiency}
\textit{Einstein} is powered by the \textit{Sheridan} drive, a closely guarded secret of the Callistan Navy. The drive is described as:

\hspace{2cm} \textit{a long sought after holy grail, a practical nuclear fusion engine}

This section will give some approximate specifications of the \textit{Sheridan} drive. Compared to other sections this section is lighter on detail and more speculative. Rather than attempt to sketch out the design of an actual nuclear fusion engine, this section gives an estimate of the efficiency of a nuclear fusion engine.

There are several choices of fuels for a nuclear fusion engine. The most likely are a deuterium tritium mix:
$$^2_1D^+ + ^3_1T^+ \rightarrow ^4_2He^{2+} (3.5 MeV) + n^0 (14.1 MeV)$$

or a helium-3 deuterium mix:

$$^2_1D^+ + ^3_2He^{2+} \rightarrow ^4_2He^{2+} (3.6 MeV) + p^+ (14.7 MeV)$$

Although the deuterium tritium reaction takes less energy to ignite, the helium-3 and deuterium mix is likely more favorable for a spacecraft engine. The byproducts of the second reaction are both charged and can be directed out of the engine bell by electromagnetic fields. The deuterium tritium reaction produces a neutral neutron which will be uniformly emitted in all directions. The neutron will also embed itself into the wall of the reactor, possibly embrittling and damaging the reactor.

The efficiency of an engine is related to its exhaust velocity. The exhaust velocity can be calculated from the kinetic energy:

$$E (joules) = \frac{1}{2}mv^2$$

Our energy units are in electron volts so the expression for velocity is:

$$v = \sqrt{\frac{2 E \frac{joules}{eV}}{m}}$$

for the helium-3 deuterium reaction this yields an exhaust velocity of:

$^4_2He^{2+}: 13,176 km/s$ or 0.044 c\\
$p^{+}: 53,059 km/s$ or 0.177 c\\

The exhaust of the \textit{Sheridan} drive is composed of two streams of particles with different exhaust velocities. The proton has an exhaust velocity of nearly 18\% the speed of light while the heavier alpha particle has a velocity of over 4\% the speed of light. The momentum change from the two exhaust streams is equivalent to the same mass being uniformly ejected at a velocity of:

$$p_{total} = 0.2mv_{p} + 0.8mv_{He} = mv_{combined}$$
$$v_{combined} = 0.2v_{p} + 0.8v_{He}$$
$$v_{combined} =  21,153km/s$$

which is equivalent to 7\% the speed of light. A common descriptor for the efficiency of an engine is its specific impulse, defined as the change in momentum per unit of fuel consumed.  When fuel quantity is expressed by mass, specific impulse is equivalent to exhaust velocity with units of meters per second. If fuel quantity is expressed by weight, then the units are expressed in seconds. The higher the exhaust velocity the greater the efficiency, and as we will see in the following section the faster a spacecraft can travel.

Since the \textit{Sheridan} drive is a first generation nuclear fusion engine we assume that its specific impulse is much less than the theoretical maximum. If we assume the specific impulse (and exhaust velocity) is an order of magnitude lower than the ideal, then the specific impulse of the Sheridan drive is:

$$I_{sp} \approx 2,100 km/s \approx 214,000 seconds$$

Going forward we will round the \textit{Sheridan} drive's specific impulse to \textbf{200,000 seconds} and the exhaust velocity to \textbf{2,000} km/s.

The table below gives a comparison of the specific impulse of existing engines.

\begin{center}
\begin{tabular}{|m{5 cm}| m{5 cm}|} \hline
\textbf{Engine} & \textbf{Specific impulse (s)}\\ \hline
Falcon 9 Merlin-1D &  311 \\ \hline
Space shuttle main engine &  453 \\ \hline
Dawn ion engine   &  3100\\ \hline
\textit{Sheridan} drive & 200,000\\ \hline
\end{tabular}
\end{center}

The specific impulse of the \textit{Sheridan} drive is nearly two orders of magnitude greater than that of the \textit{Dawn} ion engines. We will see in the next section that the specific impulse is a strong driver of $\Delta v$, which in turn determines how long it takes \textit{Einstein} to reach the sun's focal point.

% Einstein flight time
\section{\textit{Einstein} flight time}
\label{sec:flighttime}
In the story \textit{Einstein} takes ``decades" to reach the sun's focal point. This section calculates how long the trip takes based on the specific impulse of the \textit{Sheridan} drive and the distance to the focal point. Section 4 is more technically challenging than the first three sections and includes descriptions of several numerical approximations used to calculate \textit{Einstein}'s flight time.

The $\Delta v$ budget of a spacecraft is determined by the Tsiolkovsky rocket equation:

$$\Delta v = v_e ln \frac{m_0}{m_f}$$

where:

$v_e$ is exhaust velocity\\
$m_0$ is the spacecraft starting mass\\
$m_f$ is the ending mass after all fuel is expended\\

The $\Delta v$ strongly depends on the fraction of the spacecraft taken up by fuel. For \textit{Einstein} we assume that the exhaust velocity is 2,000 km/s. If the fraction of the starting mass taken up by fuel is 10\%, then the $\Delta v$ would be: 

$$\Delta v = 2,000,000\frac{m}{s} \times ln \frac{1}{0.9}$$
$$\Delta v \approx 210 km/s$$ 


For comparison, NASA's \textit{Cassini-Huygens} probe was more than 50\% fuel at launch. Since \textit{Einstein} is a first generation nuclear fusion engine with a heavy reactor, we'll assume it has an even lower fuel ratio and that it's $\Delta v$ budget is a more conservative \textbf{100 km/s}. In other words, on its flight to the focal point \textit{Einstein} can change its velocity by 100 km/s.

The flight to the focal point can be broken into several phases:
\begin{enumerate}
\item Jupiter approach burn
\item Jupiter escape burn (75,600 km from Jupiter)
\item Coast to perihelion 
\item Perihelion burn (0.1 AU from the sun)
\item Coast to focal point
\end{enumerate}

Each burn requires a certain $\Delta v$ to execute. In the following subsections we will add up how much time each of the phases takes and how much of the $\Delta v$ budget is expended. We will see that the majority of the $\Delta v$ budget is spent on the perihelion burn. For the calculations we will assume that the Jupiter escape burn takes place 75,600km from the center of Jupiter and that the perihelion burn takes place 0.1 AU from the sun. In later sections, these assumptions will be justified based on flight profiles of actual spacecraft.

% Jupiter approach burn
\subsection{Jupiter approach burn}
In the story \textit{Einstein} is launched from Callisto first into a parking orbit around the moon. Afterwards, the telescope is launched on a trajectory that brings it close to Jupiter. Figure~\ref{fig:jupiterapproach} illustrates the trajectory \textit{Einstein} takes.

\begin{figure}[H]
	\makebox[\textwidth][c]{\includegraphics[width=3.8in]{jupiter_approach.png}}
	\caption{Illustration of \textit{Einstein}'s orbit around Jupiter after launching from Callisto. The launch vehicle puts the telescope on a trajectory that brings it close to Jupiter (figure not to scale).}
	\label{fig:jupiterapproach}
\end{figure}

For several reasons we will not include the $\Delta v$ of the Jupiter approach burn or the time it takes to reach Jupiter into our calculations.

The transit time from Callisto to Jupiter is on the order of several days to a few weeks (for comparison Callisto's orbital period is 17 days). Since the time needed to escape from Jupiter is relatively small compared to the years \textit{Einstein} spends coasting to the sun we will ignore the time when calculating total flight time.

In the story the rocket that launches \textit{Einstein} also sends it on a trajectory that takes it close to Jupiter. Since \textit{Einstein} isn't expending any fuel to make its dive toward Jupiter, we will not include the $\Delta v$ of the Jupiter approach burn in our calculations.

$\Delta$ \textbf{v:} 0 km/s\\
\textbf{Time:} 0

% Jupiter escape burn
\subsection{Jupiter escape burn}
The second leg of the trip is to escape from Jupiter's gravity well. After \textit{Einstein} is launched from Callisto, it flies on an elliptical orbit that takes it near Jupiter. When the craft is near Jupiter the \textit{Sheridan} drive performs its first burn to escape from Jupiter's gravity well and put it on a trajectory close to the sun. Figure~\ref{jupiterescape} illustrates \textit{Einstein}'s trajectory as it executes the Jupiter escape burn. Following this burn, the spacecraft is placed in a highly eccentric orbit with a perihelion close to the sun and an aphelion at Jupiter's orbit.

\begin{figure}[H]
	\makebox[\textwidth][c]{\includegraphics[width=5in]{jupiter_escape.png}}
	\caption{Illustration of \textit{Einstein}'s orbit around Jupiter. When the telescope is at its closest approach the \textit{Sheridan} performs the Jupiter escape burns to send the spacecraft on a trajectory leaving Jupiter and into a heliocentric orbit.}
	\label{jupiterescape}
\end{figure}

Based on the methods discussed in the \hyperref[subsec:diving]{first section}, we can calculate the speed of a spacecraft with a perihelion at 0.1 AU and an aphelion at 5 AU (Jupiter's orbit). At the aphelion, a spacecraft with this orbit will be traveling at 2.6 km/s. Jupiter's orbital velocity averages 13.1 km/s. Thus, \textit{Einstein} will need to be traveling 10.5 km/s relative to Jupiter, in the direction opposite to Jupiter's orbit.

This means that when \textit{Einstein} is performing the Jupiter escape burn it must have a $v_{\infty}$ relative to Jupiter of 10.5km/s. From section 1 we know the expression for $v_{\infty}$ is:

$$v_{\infty} = \sqrt{(v_p + \Delta v_2)^2-\frac{2MG}{r_p}}$$

Where:
$v_0$: starting velocity\\
$\Delta v$: change in velocity supplied by \textit{Sheridan} drive\\
$r_p$: starting distance from Jupiter

We assume that $r_p$ is 75,600km, the same distance from the center of Jupiter that NASA's \textit{Juno} probe will reach on its closest approach. Knowing that \textit{Einstein} started at Callisto's orbit, we can calculate $v_p$ using the equation for specific orbital energy seen earlier:

$$\frac{1}{2}v^2-\frac{MG}{r} = -\frac{MG}{2a}$$

where:\\
$a: r_p + r_a$
$r = r_p: 7.56\times 10^7m$\\
$r_a:$ orbital distance of Callisto $1.88 \times 10^9$m\\
$M:$ mass of Jupiter $1.90\times10^27$\\


substituting gives us \textbf{$v_p = 57.3 km/s$}

Plugging the values for $v_p$, $v_{\infty}$, and $r_p$ into the equation for $v_{\infty}$ above gives us $\Delta v$ \textbf{= 1.56km/s} for the Jupiter escape burn.

It only takes a small $\Delta v$ to escape from Jupiter because \textit{Einstein} starts from a highly eccentric orbit. Burning at perijove (the point closest to Jupiter) takes advantage of the Oberth effect and allows \textit{Einstein} to have a high $v_{\infty}$ while using little fuel.

Like in the previous section, we will ignore the time it takes \textit{Einstein} to escape from Jupiter's gravity well because the time is relatively short compared to the rest of the flight. 


$\Delta$ \textbf{v:} 1.6 km/s\\
\textbf{Time:} 0

% subsubsection
\subsection{Coast to perihelion}
The next leg of the trip is when \textit{Einstein} coasts from Jupiter inwards towards the sun. The spacecraft follows an elliptical orbit and does not expend any fuel. We will use Kepler's law to calculate how long it takes \textit{Einstein} to coast from Jupiter's orbit to its closest approach to the sun. The period of an orbit is given by Kepler's second law:

$$T^2 \propto a^3$$

which states that the period of an orbit squared is proportional to the semi major axis cubed. The earth has a period of one year and a semi-major axis of almost exactly one AU. Using the units of years and AU the ratio between period squared and semi-major axis cubed is one.

The elliptical orbit that \textit{Einstein} takes has a semi-major axis of:

$$a = \frac{1}{2}(r_{perihelion} + r_{aphelion})$$

where\\
$r_{perihelion}:$ closest approach to the sun = 0.1 AU\\
$r_{aphelion}:$ furthest distance to the sun = 5 AU\\

The perihelion is limited by how close \textit{Einstein} can be to the sun without overheating. For comparison NASA's Parker Solar Probe, which has specially designed thermal protection, will fly within 0.046 AU of the sun. Substituting into Kepler's law gives a period of:

$$1 = \frac{T^2}{a^3}$$
$$2.05AU^{3} = T^2$$
$$T = 2.93 years$$

The period of \textit{Einstein}'s elliptical trajectory is about 2.93 years. The time it takes \textit{Einstein} to complete half an orbit (starting from the aphelion and ending at perihelion) is half that, or about 1.5 years.

$\Delta$ \textbf{v:} 0 km/s\\
\textbf{Time:} 1.5 years

% Perihelion burn
\subsection{Perihelion burn}
The burn at perihelion is the \textit{Sheridan} drive's longest and most powerful burn. In the story the engine burns continuously for 10 days, accelerating \textit{Einstein} out of the solar system. In this section we'll calculate \textit{Einstein}'s $v_{\infty}$ after performing the burn. We're going to make two approximations that will make calculations much easier while giving us a reasonably accurate answer:

\begin{enumerate}
\item \textit{Einstein}'s trajectory is a straight line that starts at the perihelion and continues perpendicular to a line connecting the perihelion to the sun.
\item The long, multi day burn is approximated as a series of discrete, instantaneous velocity changes (step burns) separated by periods when \textit{Einstein} cruise. During the cruise phases \textit{Einstein}'s velocity is governed only by the effect of gravity.
\end{enumerate}

In reality, \textit{Einstein} follows a hyperbolic path that constantly changes as the burn continues. The approximation underestimates the final $v_{\infty}$ because the straight path moves away from the sun more quickly than a hyperbolic path that curves slightly inwards towards the sun. Leaving the sun's gravity well more quickly reduces the Oberth effect, and leads to a lower $v_{\infty}$. However, it is still a \textit{fairly} good approximation. Figure~\ref{fig:suntrajectory} below illustrates how the approximated trajectory compares to a more \textit{Einstein}'s actual path.

\begin{figure}[H]
	\makebox[\textwidth][c]{\includegraphics[width=5in]{solar_escape.png}}
	\caption{Comparison of \textit{Einstein}'s actual trajectory and the approximated trajectory. Figure not drawn to scale.}
	\label{fig:suntrajectory}
\end{figure}

In our approximation, \textit{Einstein}'s velocity will have a sawtooth pattern as it flies away from the sun. Each burn is equally spaced in time and increases the spacecraft's velocity instantaneously. During the cruise periods \textit{Einstein}'s velocity falls off slightly as it flies away from the sun.

\begin{figure}[H]
	\makebox[\textwidth][c]{\includegraphics[width=5in]{einstein_velocity.png}}
	\caption{Illustration of \textit{Einstein}'s approximated velocity as it flies away from the sun. Instead of constantly accelerating, in our approximation we instantaneously increase the telescope's velocity through a series of step burns. The combined $\Delta v$ of the step burns is equivalent to the $\Delta v$ of the perihelion burn. Between the burns the velocity drops as \textit{Einstein} coasts away from the sun. The time between each step burn is kept constant, resulting in the distance between each step burn increasing as the velocity increases. Graph not drawn to scale.}
	\label{fig:sawtooth}
\end{figure}

We will use the following steps to approximate the telescope's velocity after it's perihelion burn:

\begin{enumerate}
\item Divide the $\Delta v$ of the perihelion burn by the number of steps to get the $\Delta v$ of each \textbf{step burn}.
\item Divide the how long the perihelion burn is by the number of steps to get the \textbf{step cruise time}. 
\item Instantaneously increase the velocity by the $\Delta v$ of the step burn.
\item Use a linear approximation to estimate how far \textit{Einstein} should coast for so that the coast time is equal to the step cruise time.
\item Update \textit{Einstein}'s position and velocity after the cruise phase as it coasts on a straight line. The time spent during each cruise phase between the step burns should equal the step cruise time.
\item Repeat steps 2-5 until the burn and the total velocity change is complete. This breaks the continuous, multi-day burn into many small burns.
\end{enumerate}

The first three steps are straightforward - we're calculating how large each step burn is and how long the cruise step time between each burn should be to maintain a constant average acceleration.

Step 4 is the most sophisticated step. The goal of this step is to calculate \textit{how far} \textit{Einstein} needs to coast during each cruise period so that the \textit{time it takes to coast that distance} is equal to the step cruise time. Since \textit{Einstein}'s velocity changes as it flies away from the sun, the distance it travels during each cruise phase also changes. Calculating the distance traveled during each cruise phase let's us update \textit{Einstein}'s position and velocity before the next step burn.

% creating linear approximation
\subsubsection{Creating a linear approximation for step 4}
The goal of step 4 is to find what \textit{Einstein}'s new position and velocity are after it coasts for the step cruise time following a step burn.

The relation between cruise time and cruise distance is given by: 

$$T = \int_{x_0}^{x_f} \frac{1}{v(x)} dx$$

where:\\
$x_0$: starting distance traveled from perihelion\\
$x_f$: ending distance traveled from perihelion\\
$v(x)$: velocity as a function of distance from perihelion\\
$T$: time elapsed - cruise step time\\

The cruise distance is equal to $x_f - x_0$.

If we can solve for this integral analytically, we can rearrange it and get an explicit formula for the cruise distance as a function of time. Updating \textit{Einstein}'s position is then simply plugging in the desired cruise time between burns and calculating its new position. However, as we will see this integral cannot be solved analytically.

The equation for velocity as a function of distance from the sun $r$ comes from the equation for specific orbital energy. Since the telescope is coasting, the sum of its kinetic and potential energy is constant:

$$\frac{1}{2}v_0^2-\frac{M_sG}{r_0} = \frac{1}{2}v(r)^2-\frac{M_sG}{r}$$
$$v(r) = \sqrt{2  (\frac{1}{2}v_0^2-\frac{M_sG}{r_0} + \frac{M_sG}{r})}$$
$$v(r) = \sqrt{v_0^2-\frac{2M_sG}{r_0} + \frac{2M_sG}{r}}$$

This expression gives the velocity as a function of $r$ the distance from the sun. However, we need an expression for velocity as a function of $x$, the distance traveled from the perihelion. 

Using the relation:

$$r^2 = x^2 + r_p^2$$

where $r_p$ is the perihelion distance gives us:

$$v(x) = \sqrt{v_0^2-\frac{2M_sG}{\sqrt{x_0^2 + r_p^2}} + \frac{2M_sG}{\sqrt{x^2 + r_p^2}}}$$

Substituting this equation into the equation for time gives an integral that cannot be solved analytically. Instead, we can use a linear approximation for the velocity as a function of distance:

$$v(x) \approx v_0 + v'(x_0)x$$

Substituting the linear approximation into the time integral gives:

$$\boxed{T = \int_{x_0}^{x_f} \frac{1}{v_0+v'(x_0)x} dx}$$

where:\\
$v_0$= velocity at the beginning of the cruise phase\\
$x_0$: starting distance traveled from perihelion\\
$x_f$: ending distance traveled from perihelion\\
$v(x)$: velocity as a function of distance from perihelion\\
$T$: time elapsed - cruise step time\\

Going forward, we will solve this integral and use a linear approximation for \textit{Einstein}'s velocity.

% cruise time as function of distance 
\subsubsection{Solving for cruise distance as a function of time}

Now that we have the form of a linear approximation for velocity, we need to create an expression for the cruise distance $(x_f-x_0)$ as a function of time ($T$).

Using $m$ for the derivative of velocity at $x_0$:

$$m = \frac{dv}{dx}(x_0)$$

we can solve for the integral using $u$ substitution:

$$T = \int_{x_0}^{x_f} \frac{1}{v_0+mx} dx$$
$$u = v_0+mx\hspace{10mm}du = mdx\hspace{10mm}dx = \frac{1}{m}du$$
$$T = \int \frac{1}{u} \frac{1}{m}dx$$
$$T = \frac{1}{m} \log{u}$$
$$T = \frac{1}{m} \log{(v_0+mx)}\bigg]_{x_0}^{x_f}$$
$$T = \frac{1}{m} \bigg(\log{(v_0+mx_f)} - \log{(v_0+mx_0)}\bigg)$$

We want an expression for the \textbf{final position $(x_f)$} of \textit{Einstein} after it cruises for a given step cruise time  $(T)$ so we rearrange the equation:

$$mT = \log{(v_0+mx_f)} - \log{(v_0+mx_0)}$$
$$\log{(v_0+mx_f)} = \log{(v_0+mx_0)} + mT$$
$$v_0 + mx_f = e^{\log{(v_0+mx_0)}+mT}$$
$$v_0 + mx_f = e^{mT}(v_0+mx_0)$$
$$mx_f = e^{mT}(v_0+mx_0) - v_0$$
$$\boxed{x_f = \frac{1}{m}\bigg(e^{mT}(v_0+mx_0) - v_0\bigg)}$$

The only piece of this equation we're missing is an expression for $m$, the slope of the velocity as a function of distance traveled from the perihelion $x$. 

% derivative of v(x0) 
\subsubsection{Calculating $v'(x_0)$}

For the equation in the above section we need to solve for $m$, which we've defined as $v'(x_0)$. We can solve for this by taking the derivative of $v(r)$ with respect to $r$ and applying the chain rule:

$$\frac{dv}{dx} = \frac{dv}{dr}\times\frac{dr}{dx}$$

From an earlier section we derived the expression for $v(r)$:

$$v(r) = \sqrt{v_0^2-\frac{2M_sG}{r_0} + \frac{2M_sG}{r}}$$

Solving for $\frac{dv}{dr}$:

$$\frac{dv}{dr} = \frac{1}{2} \bigg(v_0^2-\frac{2M_sG}{r_0} + \frac{2M_sG}{r}\bigg)^{-\frac{1}{2}}\frac{dv}{dr}\bigg(v_0^2-\frac{2M_sG}{r_0} + \frac{2M_sG}{r}\bigg)$$

$$\frac{dv}{dr} = \frac{1}{2} \bigg(v_0^2-\frac{2M_sG}{r_0} + \frac{2M_sG}{r}\bigg)^{-\frac{1}{2}}2M_sG\log{r}$$

$$\frac{dv}{dr} = \frac{2M_sG \log{r}}{2 \sqrt{v_0^2-\frac{2M_sG}{r_0} + \frac{2M_sG}{r}}}$$

$$\frac{dv}{dr} = \frac{M_sG \log{r}}{ \sqrt{v_0^2-\frac{2M_sG}{r_0} + \frac{2M_sG}{r}}}$$

The expression for $\frac{dv}{dr}$ quickly becomes complicated and we still need to use the chain rule to reach $\frac{dv}{dx}$. Instead of explicitly calculating the derivative, we can approximate the derivative by calculating the velocity at two points and solving for the slope:

$$\frac{dv}{dx} \approx \frac{v(x_0 +\Delta) - v(x_0)}{\Delta}$$

Using this equation and plugging in a small $\Delta$ will give us a good estimate for the slope. At the beginning of the cruise phase we know the initial velocity, $v(x_0)$. We can calculate the velocity at the second point $v(x_0 + \Delta)$ by using conservation of specific orbital energy, expressed by our equation for $v(r)$:

$$v(r) = \sqrt{v_0^2-\frac{2M_sG}{r_0} + \frac{2M_sG}{r}}$$

and using the relation between $r$, $x$, and $r_p$:

$$r^2 = x^2 + r_p^2$$

% final expression for approximation
\subsubsection{Calculating the result}
To recap, the goal of creating a linear approximation was to estimate  \textbf{how far \textit{Einstein} coasts} during the set step cruise time. Knowing the cruise distance lets us update the spacecraft's position and velocity at the end of each cruise  phase. To solve for time we used two steps:
\begin{enumerate}
\item Calculate the slope of the velocity as a function of distance traveled immediately after the step burn
\item Plug the time, the slope, the initial velocity, and the initial position into the expression for the final position:

$$\boxed{x_f = \frac{1}{m}\bigg(e^{mT}(v_0+mx_0) - v_0\bigg)}$$

We arrived at this solution by rearranging the integral:
$$\boxed{T = \int_{x_0}^{x_f} \frac{1}{x_0+v'(x_0)x} dx}$$
\end{enumerate}

We repeat this process until the sum of the step burns equals the $\Delta v$ of the total perihelion escape burn. 

Using the following inputs:

\begin{center}
\begin{tabular}{|m{4 cm}| m{4 cm}| m{4 cm}|} \hline
\textbf{Parameter} & \textbf{Value} & \textbf{Units}\\ \hline
n & 1000& steps\\ \hline
$\Delta v$ & 100& km/s \\ \hline
$r_p$      &  0.1& AU\\ \hline
$v_p$     &  132 & km/s\\ \hline
Perihelion burn time & 10 & days \\ \hline
Step burn $\Delta v$ & 100 & m/s\\ \hline
Step cruise time & 864& seconds \\ \hline
\end{tabular}
\end{center}

yields the final orbital parameters for \textit{Einstein} after its perihelion burn. Note that since the Jupiter escape burn required a relatively small $\Delta v$, we set the perihelion escape burn as 100 km/s for a total $\Delta v$ budget of 101.6 km/s. 

\begin{center}
\begin{tabular}{|m{4 cm}| m{4 cm}| m{4 cm}|} \hline
\textbf{Parameter} & \textbf{Value} & \textbf{Units}\\ \hline
Position (x) & 0.72 & AU from perihelion\\ \hline
Radial distance (r) & 0.73& AU from the sun\\ \hline
Velocity      &  174 & km/s\\ \hline
$v_{\infty}$     &  167& km/s\\ \hline
\end{tabular}
\end{center}

A 100 km/s escape burn spread over 10 days (an average acceleration of $0.12m/s^2$) increases \textit{Einstein}'s velocity to 174km/s. At the end of the burn the spacecraft is 0.73 AU from the sun, or just past the orbit of Venus.


$\Delta$ \textbf{v:} 100 km/s\\
\textbf{Time:} 10 days\\

% coast to focal point
\subsection{Coast to focal point}
After its the perihelion burn, \textit{Einstein} makes an unpowered coast to the focal point. This is the longest leg of the telescope's journey. As the spacecraft flies further from the sun it slows down, exchanging kinetic energy for greater potential energy. The velocity of \textit{Einstein} as a function of its distance from the sun $r$ is equivalent to its total specific orbital energy at the end of its perihelion burn:

$$E_{b} = \frac{1}{2}v_b^2 - \frac{MG}{r_b} = \frac{1}{2}v^2-\frac{MG}{r}$$
$$\frac{1}{2}v^2 = E_b+\frac{MG}{r}$$
$$\boxed{v = \sqrt{2\big(E_b + \frac{MG}{r}\big)}}$$

Where:

$E_b$ is the specific orbital energy after the perihelion burn\\
$v_b$ is the spacecraft velocity after the perihelion burn\\
$r_b$ is the spacecraft distance from the sun after the perihelion burn\\

To calculate \textit{Einstein}'s flight time we again need to solve the integral:

$$T = \int_{x_0}^{x_f} \frac{1}{v(x)} dx$$

This is the reverse of the problem we had in the previous section. In the last section we needed to solve for a distance that \textit{Einstein} travels during the step cruise time. Now we are solving for the time it takes \textit{Einstein} to coast a certain distance - to the suns focal point. We will evaluate this integral where:

$x_0 = x_b$: spacecraft distance from the perihelion after the perihelion burn\\
$x_f$: focal distance of the sun\\
$v(x)$ the expression for velocity. Note that the expression is for velocity as a function of $x$, rather than for $r$ in the expression above. $x$ and $r$ are related through:

$$r^2 = x^2 + r_p^2$$

As we saw in the previous section, this integral cannot be solved analytically. Instead, we use Simpson's rule to approximate:

$$\int_a^b f(x) dx \approx \frac{\Delta x}{3} (f(x_0) + 4f(x_1)+ 2f(x_2) + 4f(x_3) + 2 f(x_4) + ... + 4f(x_{n-1}) + f(x_n))$$


Using a step size $\Delta x$ of 0.55AU we get a flight time of 15.7 years from $r_b$ to the focal point (550 AU). Adding the time \textit{Einstein} spent coasting from Jupiter to the perihelion (1.5 years) gives us a \textbf{total estimated flight time of about \textbf{17 years}}. This is the time it takes for \textit{Einstein} to just reach the beginning of the sun's focal point - as it continues to coast it stays on the sun's focal line where the magnification steadily becomes stronger. For comparison, NASA's New Horizon's probe spent 9 years reaching Pluto (33 AU).

% subsection - additional considerations
\subsection{Additional considerations}
Several major simplifications were made when calculating \textit{Einstein}'s flight time. One of the most prominent is that orbital inclination was ignored.

We calculated the time it would take \textit{Einstein} to reach the sun's focal point. However, to actually observe a planet a gravitationally lensing telescope can't fly in any direction - it must reach the sun's focal point \textit{and} be in line with the sun and the target planet. If the target planet is not aligned with the ecliptic - a plane in space defined by the Earth's orbit - then a spacecraft must perform an additional burn that was not included. The added burn changes the spacecraft's orbital inclination, the angle between the plane of the orbit and the ecliptic. 

Changing the orbital inclination sends a spacecraft either above or below the plane of planets orbiting the sun, but requires more $\Delta v$. However, changing the orbital inclination does not require much $\Delta v$ for a spacecraft traveling slowly. When \textit{Einstein} exits from Jupiter's orbit it is traveling at just 2.6 km/s relative to the sun. Changing the spacecraft's orbital inclination should thus require only a few km/s of $\Delta v$.

\section{Code}
Several Python programs were written to perform numerical approximations and generate graphs. All code used in this appendix can be found at:

\url{https://github.com/timsliu/flybys_foci}

\section{References}

Much of the information in the appendix can be found at the corresponding Wikipedia pages. Several other sources that were used include:

Landis Geoffrey. \textit{Mission to the Gravitational Focus of the Sun: A Critical Analysis}. NASA John Glenn Research Center. 2016.

Widnall J. Peraire. \textit{Lecture L16 - Central Force Motion: Orbits}. MIT 16.07 Dynamics. Fall 2008.

Turyshev Slava, Shao Michael. \textit{Using the Sun as a Cosmic Telescope}. Scientific American. 30 May 2017.


\end{document}
